\chapter{Primitive Types}

\section{Numbers}

% everything is a number, not always obvious, classes of numbers, the simples calls is integers
All values that we may want to work with in a computer are, one way or the other, \idx{numbers}{Number}. This fact may not always be obvious, but it will be quite a while before you encounter such examples. From public school and highschool, we know these classes of numbers:
\begin{itemize}
  \descitem{The \idx{natural numbers}{Numbers!Natural}} ($\mathbb{N}$) These are the non-negative numbers that can be written without a decimal point\footnote{There are two definitions of natural numbers. The other definition is that they are the positive numbers that can be written without a decimal point. That is to say, without zero. As this definition is generally less applicable in computers, we stick to the first definition.}.
  \descitem{The \idx{integers}{Numbers!Integer}} ($\mathbb{Z}$) These are all the natural numbers (according to the first definition) along with the negative versions of the natural numbers (according to the second definition). That definition juggling is to make sure that we don't have a negative zero.
  \descitem{The \idx{real numbers}{Numbers!Real}} ($\mathbb{R}$) These are all the numbers that can be placed on a line between $-\infty$ and $\infty$. We often refer to these as floating point numbers.
  \descitem{The \idx{rational numbers}{Numbers!Rational}} ($\mathbb{Q}$) These are then numbers that can be written as the fraction $m/n$ where $m \in \mathbb{Z}$ and $m \in \mathbb{N}$.
  \descitem{The \idx{irrational numbers}{Numbers!Irrational}} ($\mathbb{P}$) These are numbers that belong to $\mathbb{R}$ but not $\mathbb{Q}$. This is where $\pi$ and $e$ belong.
\end{itemize}
Computers typically works with other classes of numbers. In the following sections, we will explore the most commonly used ones. Common to these is that they are all \idx{represented in binary}{Representation!Binary} numbers. That is, using the \idx{base-2}{Base 2} digits. We are used to working with \idx{base-10}{Base 10} where numbers are digits between zero and nine. In base-2, the numbers are made up of digits between zero and one (i.e., \texttt{0} or \texttt{1}). Such a digit is called a \defi{bit}{Bit}, and a sequence of eight bits is referred to as a \defi{byte}{Byte}. The plural forms of the singular bit and byte is bits and bytes. Bits are, by the way, the \idx{SI unit}{SI unit} of information, in the same way as meter is the SI unit of distance. Accordingly, we measure \idx{information}{Information} in bits.

\section{Integers}

\subsection{Operations}

% basic operations
All general purpose languages support the four basic \idx{arithmetic operators}{Operation!Arithmetic} for addition, subtraction, multiplication and division of integers. The tree first of these will guarantee an integral result. For instance, if you add two integers, you will get an integer. The \textsl{size} of this integer may be affected though. But more on that in section \ref{primitives:int:representation}.

% that division thing
That rule, about the result of an operation resuling in an integral value does not hold for the \idx{division}{Operation!Division} operation though. Mathematically, the result \textsl{may} be integral, but generally speaking there is no guarantee, and more often than not it is simply not the case. Most programming languages supports two types of division; namely \idx{integer division}{Operation!Division!Integer} and \idx{floating-point division}{Operation!Division!Floating-point}. The result of a floating-point division is a floating-point number as introduced in section \ref{primitives:float}. The result of an integer division is an integer.

% remainder vs modulo
This operation is tightly coupled to both the \idx{remainder}{Remainder} and the \idx{modulo}{Modulo} operations. The difference between these is how they handle negative numbers.

% TODO: Figure of remainder vs modulo

\subsection{Representation}
\label{primitives:int:representation}

% typical representation

% signedness

\csharpsubsection{\csharp}

\begin{syntaxsegment}
  Test
\end{syntaxsegment}

\begin{syntaxfloat}
  \begin{syntax}{expr}
  \SyntaxWestSplit{MainWest}
  \SyntaxEastSplit{MainEast}
  
  \node[nonterminal] (ruleIa)  at \SyntaxDistribute{MainWest}{MainEast}{1}{3} {expr};
  \node[terminal]    (ruleIb)  at \SyntaxDistribute{MainWest}{MainEast}{2}{3} {+};
  \node[nonterminal] (ruleIc)  at \SyntaxDistribute{MainWest}{MainEast}{3}{3} {expr};
  \node[nonterminal] (ruleIIa) at \SyntaxDistributeLine{MainWest}{MainEast}{1}{3}{1} {expr};
  \node[terminal]    (ruleIIb) at \SyntaxDistributeLine{MainWest}{MainEast}{2}{3}{1} {-};
  \node[nonterminal] (ruleIIc) at \SyntaxDistributeLine{MainWest}{MainEast}{3}{3}{1} {expr};
  \node[nonterminal] (ruleIIIa) at \SyntaxDistributeLine{MainWest}{MainEast}{1}{3}{2} {expr};
  \node[terminal]    (ruleIIIb) at \SyntaxDistributeLine{MainWest}{MainEast}{2}{3}{2} {*};
  \node[nonterminal] (ruleIIIc) at \SyntaxDistributeLine{MainWest}{MainEast}{3}{3}{2} {expr};
  \node[nonterminal] (ruleIVa) at \SyntaxDistributeLine{MainWest}{MainEast}{1}{3}{3} {expr};
  \node[terminal]    (ruleIVb) at \SyntaxDistributeLine{MainWest}{MainEast}{2}{3}{3} {/};
  \node[nonterminal] (ruleIVc) at \SyntaxDistributeLine{MainWest}{MainEast}{3}{3}{3} {expr};
  \node[nonterminal] (ruleVa) at \SyntaxDistributeLine{MainWest}{MainEast}{1}{3}{4} {expr};
  \node[terminal]    (ruleVb) at \SyntaxDistributeLine{MainWest}{MainEast}{2}{3}{4} {\%};
  \node[nonterminal] (ruleVc) at \SyntaxDistributeLine{MainWest}{MainEast}{3}{3}{4} {expr};
  
  \draw[path] (begin)--(ruleIa)--(ruleIb)--(ruleIc)--(end);
  \draw[path] (begin) to[-|-] (ruleIIa)--(ruleIIb)--(ruleIIc) to[-|-] (end);
  \draw[path] (begin) to[-|-] (ruleIIIa)--(ruleIIIb)--(ruleIIIc) to[-|-] (end);
  \draw[path] (begin) to[-|-] (ruleIVa)--(ruleIVb)--(ruleIVc) to[-|-] (end);
  \draw[path] (begin) to[-|-] (ruleVa)--(ruleVb)--(ruleVc) to[-|-] (end);
\end{syntax}

  \caption{Expressions of arithmetic operators}
  \label{syntax:prim:arithmetic:ops}
\end{syntaxfloat}

% integer types available

\begin{figure}[tbp]
  \begin{center}
  \begin{tabular}{rccl}
    \emph{Size} & \emph{Unsigned Type} & \emph{Signed Type} & \emph{Comment} \\
     8 bits & \typename{byte} & \typename{sbyte} & \\
    16 bits & \typename{ushort} & \typename{short} & \\
    32 bits & \typename{uint} & \typename{int} & \\
    64 bits & \typename{ulong} & \typename{long} & \\
    1 word & \typename{nuint} & \typename{nint} & Do not use\\
  \end{tabular}
\end{center}

  \caption{Integer types available in \csharp.}
  \label{fig:prim:int:csharp:types}
\end{figure}

% what is the result of int32+int32? what is an overflow?

\elixirsubsection{Elixir}

% arbitrary sizes: pros
In Elixir, there is only one integer type and that has an \idx{arbitrary size}{Arbitrary size}. As long as an integral value fits in memory, an Elixir integer can hold it. The value is not in having integers that each take up 90\% of your \idx{primary memory}{Memory!Primary}. It is very rare that we have a need for integers beyond 128 bits or even 64 bits. But when we work with \idx{fixed size}{Fixed size} integers, we always have to be aware of that hard limit. In Elixir, we don't have to: If we somehow end up with a number needing 7000 bits, then that is what gets allocated. We don't have to worry.

% and cons, why the tradeoff was resolved in this way
Elixir code, still runs on the same hardware though. This hardware does not have instructions capable of operating on 7000 bit integers. So, instead of an integer operation being a single \idx{instruction}{Instruction}, it is an \idx{algorithm}{Algorithm} in itself. This is significantly slower. Elixir is designed for network intensive tasks, and these are even slower. So, for the tasks where one would choose Elixir, it basically doesn't matter. And that is why the designers of that language has made that choice.

\section{Floating Point Numbers}
\label{primitives:float}

\subsection{Operations}
\subsection{Representation}
\csharpsubsection{\csharp}
\elixirsubsection{Elixir}

\section{Truth Values}

% what do they represent

\subsection{Operations on Nontruthy Values}

Any valid comparison between two nontruthy values will yield a truthy value. For instance, if we ask \quoted{is 42 equal to 56?} then the resulting value is \valuename{false} but if we ask \quoted{is 1 less than or equal to 2?} then the resulting value is \valuename{true}.

\begin{figure}[tbp]
  \begin{center}
  \begin{tabular}{cl}
    $a$ \texttt{==} $b$ & True if $a$ is the same as $b$ \\
    $a$ \texttt{!=} $b$ & True if $a$ is different from $b$ \\
    $a$ \texttt{<} $b$ & True if $a$ is less than $b$ \\
    $a$ \texttt{<=} $b$ & True if $a$ is less than or equal to $b$ \\
    $a$ \texttt{>} $b$ & True if $a$ is greater than $b$ \\
    $a$ \texttt{>=} $b$ & True if $a$ is greater than or equal to $b$ \\
  \end{tabular}
\end{center}

  \caption{Comparison operators.}
  \label{fig:prim:bool:comparison}
\end{figure}

\subsection{Operations on Truth Values}

\begin{figure}[tbp]
  \begin{center}
  \begin{tabular}{cccccc}
    $a$ & $b$ & $a \wedge b$ & $a \vee b$ & $a \oplus b$ & $\neg a$\\
    0 & 0 & 0 & 0 & 0 & 1 \\
    0 & 1 & 0 & 1 & 1 & \\
    1 & 0 & 0 & 1 & 1 & 0 \\
    1 & 1 & 1 & 1 & 0 & \\
  \end{tabular}
\end{center}

  \caption{Truth table for the boolean operations.}
  \label{fig:prim:bool:and}
\end{figure}

% TODO: De Morgan's law

\subsection{Representation}

% technically a bit, but typically (mostly unless in array form) a byte or word

\csharpsubsection{\csharp}

\elixirsubsection{Elixir}

\csubsection{C}

% low-level language
\idx{C}{Language!C} is a simple \idx{low-level language}{Language!Low-level}. That means that it mirrors the fundamental properties of the underlying harware and adds some highly convenient abstractions. These abstractions are chosen is such a way that they essentially can be delivered without a performance overhead.

% consequence: a boolean is a register
\idxx{Machine code} does not have a boolean type. Instead \idx{register}{Register} values that are represented using all zeroes in binary are \textsl{false} and every other value is \textsl{true}. That means -- in terms of integers -- that zero is \textsl{false} and non-zero is \textsl{true}. % TODO: Explain the !!42 == 1 situation, word for reference/strong true values, implicit konvertering til bool i condition af en if

% consequence: a bool is an integer and can thus be used in an integer expression
A consequence of this is that C doesn't have a notion of a \idx{boolean}{Boolean}. Instead, integers are used: A boolean is an interpretation of an integer, and can thus be used in an integer expression. Typically, this will make absolutely no difference. Proponents of \csharp\ will point out that booleans and integers are fundamentally different notions and should thus be treated differently. Proponents of C will point out that this allows them to write code such as this:

% TODO: Example

% explanation: why is this clever (no branches gives execution speed, and it is still readable)

\section{Variables}
\csharpsubsection{\csharp}
\elixirsubsection{Elixir}

\section{Parsing}
\subsection{Operator Precedence}
\subsection{Operator Associativity}

\exercises{primitives}{Primitive Types}

