\section{The Machine}
\label{sec:machine}

\subsection{History}

% Antikythera mechanism https://en.wikipedia.org/wiki/Antikythera_mechanism

% Jacquard machine https://en.wikipedia.org/wiki/Jacquard_machine

% Difference Engine https://en.wikipedia.org/wiki/Difference_engine

% Mutual Exclusion https://en.wikipedia.org/wiki/Mutual_exclusion

% C https://en.wikipedia.org/wiki/C_(programming_language)

% Taming of the Alpaca

\subsection{Overview}

% description of computer: model description as data flow (input, memory, processor, load instruction, registers, operator instructions, ALU, store instruction, memory, output)
Figure \ref{fig:machine:computer}

\begin{figure}[tbp]
  \begin{center}
  \begin{tikzpicture}[remember picture]
    \newcommand{\spacing}[0]{ 8mm }
    \tikzstyle{edge}  = [thick,>=stealth,draw=black]
    \tikzstyle{dedge} = [thick,->,>=stealth,draw=black]
    \tikzstyle{node}=[
      overlay,
      rectangle,
      draw=purple,
      anchor=center,
      thick,
    ]
    
    \node[node,minimum height=2*\spacing,minimum width=6cm,rounded corners=4mm,dashed] (processor) at (0,0) {};
    \node[anchor=east] (processor_label) at
      (processor.west)
      {\rotatebox{90}{Processor}};
    \node[node,anchor=north west,minimum height=\spacing] (registers) at
      ([xshift=0.5*\spacing,yshift=-0.5*\spacing]processor.north west)
      {Registers};
    \node[node,anchor=north east,minimum height=\spacing] (alu) at
      ([xshift=-0.5*\spacing,yshift=-0.5*\spacing]processor.north east)
      {ALU};
    \node[node,anchor=north,minimum width=6cm] (memory) at
      ([yshift=-\spacing]processor.south)
      {Memory};
    \node[anchor=north] (input) at
      ([xshift=-1.5cm,yshift=-\spacing]memory.south)
      {Input};
    \node[anchor=north] (output) at
      ([xshift=1.5cm,yshift=-\spacing]memory.south)
      {Output};
    
    % register <-> alu
    \draw[dedge] ([yshift= 2mm]registers.east)
               --([yshift= 2mm]alu.west);
    \draw[dedge] ([yshift=-2mm]alu.west)
               --([yshift=-2mm]registers.east);
    
    % register <-> memory
    \draw[dedge] ([xshift=-4mm,yshift=-1.5*\spacing]registers.south)
               --([xshift=-4mm]                     registers.south);
    \draw[dedge] ([xshift= 4mm]                     registers.south)
               --([xshift= 4mm,yshift=-1.5*\spacing]registers.south);
%    \draw[dedge] ()--();
    
    % memory <-> i/o
    \draw[dedge] (                 input.north)
               --([yshift=\spacing]input.north);
    \draw[dedge] ([yshift=\spacing]output.north)
               --(                 output.north);
  \end{tikzpicture}
\end{center}

  \caption{Model of computer.}
  \label{fig:machine:computer}
\end{figure}

\subsection{Memory Model}

\begin{figure}[tbp]
  \begin{center}
  \begin{tikzpicture}[]
    \newcommand{\cellheight}[0]{8mm}
    \newcommand{\cellwidth}[0]{24mm}
    
    \tikzstyle{dedge} = [thick,->,>=stealth,draw=black]
    
    \tikzstyle{cell}=[
      rectangle,
      draw=purple,
      anchor=north,
      thick,
      minimum height=\cellheight,
      minimum width=\cellwidth,
    ]
    \tikzstyle{address}=[
      anchor=east,
    ]
    \tikzstyle{comment}=[
      anchor=west,
    ]
    
    \node[cell,minimum height=5*\cellheight] (dynsegment) at (0, -0*\cellheight) {};
    \node[cell,draw=none] (stacksegment) at (0, -0*\cellheight) {Stack};
    \node[cell,draw=none] (dotssegment) at (0, -2*\cellheight) {};
    \node[cell,draw=none] (heapsegment) at (0, -4*\cellheight) {Heap};
    \node[cell] (datasegment) at (0, -5*\cellheight) {Data};
    \node[cell] (textsegment) at (0, -6*\cellheight) {Text};
    \node[cell] (reservedsegment) at (0, -7*\cellheight) {Reserved};
    
    \node[address] () at (reservedsegment.south west) {$0$};
    \node[address] () at (dynsegment.north west) {\textsl{max}};
    
    \node[comment] () at ([xshift=2mm]dynsegment.east) {Dynamic program data};
    \node[comment] () at ([xshift=2mm]datasegment.east) {Static program data};
    \node[comment] () at ([xshift=2mm]textsegment.east) {Program instructions};
    \node[comment] () at ([xshift=2mm]reservedsegment.east) {Input and output};
    
    \draw[dedge] (stacksegment) -- (dotssegment);
    \draw[dedge] (heapsegment) -- (dotssegment);
    
    \draw[dedge] ([xshift=-4mm,yshift= 4mm]reservedsegment.south west)
              -- ([xshift=-4mm,yshift=-4mm]dynsegment.north west)
              node[midway,sloped,above] {address range of main memory}
    ;
  \end{tikzpicture}
\end{center}

  \caption{Memory model.}
  \label{fig:machine:memory}
\end{figure}

\subsection{Registers}

\subsection{Instructions}

\subsubsection{Memory Access}

\subsubsection{Arithmetic Operations}

\subsubsection{Choice}

\subsection{Big Picture}

The components that of the machine that we have covered can be put together around the central processing unit (i.e., the \idx{CPU}{CPU}). An implementation of a primitive RISC-V processor is illustrated in figure \ref{fig:machine:riscv}. In reality, it is much more complicated, and RISC-V is a simple \idx{processor architecture}{Architecture!Processor}.

\begin{figure}[tbp]
  \begin{center}
  \begin{tikzpicture}[remember picture]
    \newcommand{\Xpc}[0]{ 8mm }
    \newcommand{\Ycenter}[0]{ 8mm }
    
    \newcommand{\spacing}[0]{ 8mm }
    
    \tikzstyle{edge}  = [thick,>=stealth,draw=black]
    \tikzstyle{dedge} = [thick,->,>=stealth,draw=black]
    \tikzstyle{block}=[
      overlay,
      rectangle,
      draw=purple,
      anchor=center,
      thick,
    ]
    
    % program counter
    \node[block] (pc) at (\Xpc, \Ycenter) {PC};
    
    % instruction increment alu
    
    % instruction branch alu
    
    % branch mux
    
    % instruction memory
    
    % registers
    
    % load mux
    
    % alu
    
    % data memory
  \end{tikzpicture}
\end{center}

  \caption{Primitive model of a RISC-V Processor.}
  \label{fig:machine:riscv}
\end{figure}

