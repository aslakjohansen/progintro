\section{The Machine}
\label{sec:machine}

\subsection{History}

% Antikythera mechanism https://en.wikipedia.org/wiki/Antikythera_mechanism

% Jacquard machine https://en.wikipedia.org/wiki/Jacquard_machine

% Difference Engine https://en.wikipedia.org/wiki/Difference_engine

% Mutual Exclusion https://en.wikipedia.org/wiki/Mutual_exclusion

% C https://en.wikipedia.org/wiki/C_(programming_language)

% Taming of the Alpaca

\subsection{Overview}

% description of computer: model description as data flow (input, memory, processor, load instruction, registers, operator instructions, ALU, store instruction, memory, output)
Figure \ref{fig:machine:computer}

\begin{figure}[tbp]
  \begin{center}
  \begin{tikzpicture}[remember picture]
    \newcommand{\spacing}[0]{ 8mm }
    \tikzstyle{edge}  = [thick,>=stealth,draw=black]
    \tikzstyle{dedge} = [thick,->,>=stealth,draw=black]
    \tikzstyle{node}=[
      overlay,
      rectangle,
      draw=purple,
      anchor=center,
      thick,
    ]
    
    \node[node,minimum height=2*\spacing,minimum width=6cm,rounded corners=4mm,dashed] (processor) at (0,0) {};
    \node[anchor=east] (processor_label) at
      (processor.west)
      {\rotatebox{90}{Processor}};
    \node[node,anchor=north west,minimum height=\spacing] (registers) at
      ([xshift=0.5*\spacing,yshift=-0.5*\spacing]processor.north west)
      {Registers};
    \node[node,anchor=north east,minimum height=\spacing] (alu) at
      ([xshift=-0.5*\spacing,yshift=-0.5*\spacing]processor.north east)
      {ALU};
    \node[node,anchor=north,minimum width=6cm] (memory) at
      ([yshift=-\spacing]processor.south)
      {Memory};
    \node[anchor=north] (input) at
      ([xshift=-1.5cm,yshift=-\spacing]memory.south)
      {Input};
    \node[anchor=north] (output) at
      ([xshift=1.5cm,yshift=-\spacing]memory.south)
      {Output};
    
    % register <-> alu
    \draw[dedge] ([yshift= 2mm]registers.east)
               --([yshift= 2mm]alu.west);
    \draw[dedge] ([yshift=-2mm]alu.west)
               --([yshift=-2mm]registers.east);
    
    % register <-> memory
    \draw[dedge] ([xshift=-4mm,yshift=-1.5*\spacing]registers.south)
               --([xshift=-4mm]                     registers.south);
    \draw[dedge] ([xshift= 4mm]                     registers.south)
               --([xshift= 4mm,yshift=-1.5*\spacing]registers.south);
%    \draw[dedge] ()--();
    
    % memory <-> i/o
    \draw[dedge] (                 input.north)
               --([yshift=\spacing]input.north);
    \draw[dedge] ([yshift=\spacing]output.north)
               --(                 output.north);
  \end{tikzpicture}
\end{center}

  \caption{Model of computer.}
  \label{fig:machine:computer}
\end{figure}

\subsection{Memory Model}

\begin{figure}[tbp]
  \begin{center}
  \begin{tikzpicture}[]
    \newcommand{\cellheight}[0]{8mm}
    \newcommand{\cellwidth}[0]{24mm}
    
    \tikzstyle{dedge} = [thick,->,>=stealth,draw=black]
    
    \tikzstyle{cell}=[
      rectangle,
      draw=purple,
      anchor=north,
      thick,
      minimum height=\cellheight,
      minimum width=\cellwidth,
    ]
    \tikzstyle{address}=[
      anchor=east,
    ]
    \tikzstyle{comment}=[
      anchor=west,
    ]
    
    \node[cell,minimum height=5*\cellheight] (dynsegment) at (0, -0*\cellheight) {};
    \node[cell,draw=none] (stacksegment) at (0, -0*\cellheight) {Stack};
    \node[cell,draw=none] (dotssegment) at (0, -2*\cellheight) {};
    \node[cell,draw=none] (heapsegment) at (0, -4*\cellheight) {Heap};
    \node[cell] (datasegment) at (0, -5*\cellheight) {Data};
    \node[cell] (textsegment) at (0, -6*\cellheight) {Text};
    \node[cell] (reservedsegment) at (0, -7*\cellheight) {Reserved};
    
    \node[address] () at (reservedsegment.south west) {$0$};
    \node[address] () at (dynsegment.north west) {\textsl{max}};
    
    \node[comment] () at ([xshift=2mm]dynsegment.east) {Dynamic program data};
    \node[comment] () at ([xshift=2mm]datasegment.east) {Static program data};
    \node[comment] () at ([xshift=2mm]textsegment.east) {Program instructions};
    \node[comment] () at ([xshift=2mm]reservedsegment.east) {Input and output};
    
    \draw[dedge] (stacksegment) -- (dotssegment);
    \draw[dedge] (heapsegment) -- (dotssegment);
    
    \draw[dedge] ([xshift=-4mm,yshift= 4mm]reservedsegment.south west)
              -- ([xshift=-4mm,yshift=-4mm]dynsegment.north west)
              node[midway,sloped,above] {address range of main memory}
    ;
  \end{tikzpicture}
\end{center}

  \caption{Memory model.}
  \label{fig:machine:memory}
\end{figure}

\subsection{Registers}

\subsection{Instructions}

\subsubsection{Memory Access}

\subsubsection{Arithmetic Operations}

\subsubsection{Choice}

\subsection{Big Picture}

The components that of the machine that we have covered can be put together around the central processing unit (i.e., the \idx{CPU}{CPU}). An implementation of a primitive RISC-V processor is illustrated in figure \ref{fig:machine:riscv}. In reality, it is much more complicated, and RISC-V is a simple \idx{processor architecture}{Architecture!Processor}. Lets briefly go through what is going on in this figure.

\begin{figure}[tbp]
  \centering

\rotatebox{90}{
\begin{tikzpicture}[]
  \newcommand{\Xpc}[0]{ 0mm }
  \newcommand{\Xim}[0]{ 32mm }
  \newcommand{\Xreg}[0]{ 56mm }
  \newcommand{\Xloadmux}[0]{ 65mm }
  \newcommand{\Xdm}[0]{ 80mm }
  \newcommand{\Ycenter}[0]{ 8mm }
  
  \newcommand{\spacing}[0]{ 4mm }
  \newcommand{\lineheight}[0]{ 8mm }
  
  \newcommand{\andheight}[0]{ 8mm }
  \newcommand{\andwidth}[0]{ (1.5*\andheight) }
  \newcommand{\muxwidth}[0]{ 4mm }
  \newcommand{\muxheight}[0]{ (\spacing+2*\muxwidth) }
  \newcommand{\muxinputoffset}[0]{ (\spacing/2) }
  \newcommand{\aluheight}[0]{ 24mm }
  \newcommand{\aluwidth}[0]{ 12mm }
  \newcommand{\aluindent}[0]{ 3mm }
  \newcommand{\aluinputoffset}[0]{ (\aluheight/4+\aluindent/2) }
  
  \tikzstyle{edge}  = [thick,>=stealth,draw=black]
  \tikzstyle{dedge} = [thick,->,>=stealth,draw=black]
  \tikzstyle{control} = [draw=teal]
  \tikzstyle{block}=[
    rectangle,
    draw=purple,
    anchor=center,
    align=center,
    thick,
  ]
  \tikzstyle{mux}=[
    rectangle,
    draw=purple,
    anchor=center,
    align=center,
    rounded corners=\muxwidth/2,
    minimum width=\muxwidth,
    minimum height=\muxheight,
    thick,
  ]
  \tikzstyle{point}=[
    circle,
    fill=black,
    anchor=center,
    minimum size=0.9mm,
    inner sep=0pt,
    thick,
  ]
  
  % #1: prefix
  % #2: position
  \newcommand{\buildProgramCounterBlock}[2]{
    \node[
      block,
      anchor=west,
      minimum height=40mm,
    ] (#1) at (#2) {\textbf{PC}};
    
    \coordinate (#1 in)  at (#1.west);
    \coordinate (#1 out) at (#1.east);
  }
  
  % #1: prefix
  % #2: position
  \newcommand{\buildInstructionMemoryBlock}[2]{
    \node[
      block,
      anchor=west,
      minimum width=40mm,
      minimum height=40mm,
    ] (#1) at (#2) {};
    \node[anchor=south,align=center] () at (#1.south) {\textbf{Instruction}\\\textbf{Memory}};
    
    \coordinate (#1 addr) at ([yshift=0.0*\lineheight]#1.west);
    \coordinate (#1 inst) at ([yshift=0.0*\lineheight]#1.east);
    
    \node[anchor=west] () at (#1 addr) {Address};
    \node[anchor=east] () at (#1 inst) {Instruction};
  }
  
  % #1: prefix
  % #2: position
  \newcommand{\buildRegisterBlock}[2]{
    \node[
      block,
      anchor=west,
      minimum width=40mm,
      minimum height=40mm,
    ] (#1) at (#2) {};
    \node[anchor=west] () at (#1.center) {\rotatebox{90}{\textbf{Registers}}};
    
    \coordinate (#1 data)   at ([yshift= 1.5*\lineheight]#1.west);
    \coordinate (#1 regI)   at ([yshift= 0.5*\lineheight]#1.west);
    \coordinate (#1 regII)  at ([yshift=-0.5*\lineheight]#1.west);
    \coordinate (#1 regIII) at ([yshift=-1.5*\lineheight]#1.west);
    \coordinate (#1 valI)   at ([yshift= \aluinputoffset]#1.east);
    \coordinate (#1 valII)  at ([yshift=-\aluinputoffset+\muxinputoffset]#1.east);
    
    \node[anchor=west] () at (#1 data)   {Data};
    \node[anchor=west] () at (#1 regI)   {Register 1};
    \node[anchor=west] () at (#1 regII)  {Register 2};
    \node[anchor=west] () at (#1 regIII) {Register 3};
    \node[anchor=east] () at (#1 valI)   {Value 1};
    \node[anchor=east] () at (#1 valII)  {Value 2};
    
    \coordinate (#1 rw) at ([xshift=-2*\spacing]#1.south east);
  }
  
  % #1: prefix
  % #2: position
  \newcommand{\buildDataMemoryBlock}[2]{
    \node[
      block,
      anchor=west,
      minimum width=40mm,
      minimum height=40mm,
    ] (#1) at (#2) {};
    \node[anchor=south,align=center] () at (#1.south) {\textbf{Data}\\\textbf{Memory}};
    
    \coordinate (#1 addr)  at ([yshift= 1.0*\lineheight]#1.west);
    \coordinate (#1 data)  at ([yshift=-1.0*\lineheight]#1.west);
    \coordinate (#1 value) at ([yshift= 0.0*\lineheight]#1.east);
    
    \node[anchor=west] () at (#1 addr) {Address};
    \node[anchor=west] () at (#1 data) {Data};
    \node[anchor=east] () at (#1 value) {Value};
  }
  
  
  % #1: prefix
  % #2: position
  \newcommand{\buildALU}[3]{
    \coordinate (offset) at (#2);
    
    \draw[
      thick,
      draw=purple,
    ]  ([yshift=\aluheight/2]offset)
    -- ([xshift=\aluwidth,yshift= \aluheight/5]offset)
    -- ([xshift=\aluwidth,yshift=-\aluheight/5]offset)
    -- ([yshift=-\aluheight/2]offset)
    -- ([yshift=-\aluindent]offset)
    -- ([xshift=\aluindent]offset)
    -- ([yshift=\aluindent]offset)
    -- cycle;
    \node[anchor=center,align=center] () at ([xshift=\aluindent+\aluwidth/2-\aluindent/2]offset) {\rotatebox{90}{\textbf{#3}}};
    
    \coordinate (#1 iI)  at ([yshift= \aluinputoffset]offset);
    \coordinate (#1 iII) at ([yshift=-1*\aluinputoffset]offset);
    \coordinate (#1 o)   at ([xshift=\aluwidth]offset);
    \coordinate (#1 c)   at ([xshift=\aluwidth/2,yshift=-(\aluheight/5/2+\aluheight/2/2]offset);
    \coordinate (#1 z)   at ([xshift=\aluwidth/2,yshift= (\aluheight/5/2+\aluheight/2/2]offset);
  }
  
  % #1: prefix
  % #2: position
  \newcommand{\buildAND}[2]{
    \coordinate (offset) at (#2);
    
    \draw[
      thick,
      draw=teal,
    ]  ([xshift=\andwidth/2,yshift= \andheight/2]offset)
    -- ([xshift=\andwidth/2,yshift=-\andheight/2]offset)
    -- ([xshift=-(\andwidth/2-\andheight/2),yshift=-\andheight/2]offset)
    arc [x radius=\andheight/2,y radius=\andheight/2, start angle=270,end angle=90]
    -- cycle;
    \node[anchor=center,align=center] () at (offset) {\rotatebox{0}{\textbf{AND}}};
    
    \coordinate (#1 iI)  at ([xshift=\andwidth/2,yshift= \spacing/2]offset);
    \coordinate (#1 iII) at ([xshift=\andwidth/2,yshift=-\spacing/2]offset);
    \coordinate (#1 o)   at ([xshift=-\andwidth/2]offset);
  }
  
  % #1: prefix
  % #2: position
  \newcommand{\buildControlBlock}[2]{
    \node[
      block,
      anchor=north west,
      minimum width=32mm,
      minimum height=7*\lineheight,
      control,
    ] (#1) at (#2) {};
    \node[anchor=south west] () at ([xshift=\spacing,yshift=\spacing]#1.south west) {\rotatebox{90}{\textbf{Control}}};
    
    \coordinate (#1 Instruction) at ([xshift= 1.0*\spacing]#1.north west);
    \coordinate (#1 RegWrite)    at ([yshift= 3.0*\lineheight]#1.east);
    \coordinate (#1 ALUSrc)      at ([yshift= 2.0*\lineheight]#1.east);
    \coordinate (#1 ALUOp)       at ([yshift= 1.0*\lineheight]#1.east);
    \coordinate (#1 MemRead)     at ([yshift= 0.0*\lineheight]#1.east);
    \coordinate (#1 MemWrite)    at ([yshift=-1.0*\lineheight]#1.east);
    \coordinate (#1 MemtoReg)    at ([yshift=-2.0*\lineheight]#1.east);
    \coordinate (#1 Branch)      at ([yshift=-3.0*\lineheight]#1.east);
    
    \node[anchor=north] () at (#1 Instruction)   {\rotatebox{90}{Instruction}};
    \node[anchor=east] () at (#1 RegWrite) {RegWrite};
    \node[anchor=east] () at (#1 ALUSrc)   {ALUSrc};
    \node[anchor=east] () at (#1 ALUOp)    {ALUOp};
    \node[anchor=east] () at (#1 MemRead)  {MemRead};
    \node[anchor=east] () at (#1 MemWrite) {MemWrite};
    \node[anchor=east] () at (#1 MemtoReg) {MemtoReg};
    \node[anchor=east] () at (#1 Branch)   {Branch};
  }
  
  % program counter
  \buildProgramCounterBlock{pc}{0,0}
  
  % instruction memory
  \buildInstructionMemoryBlock{im}{[xshift=2*\spacing]pc.east}
  
  % instruction increment alu
  \buildALU{alu incr}{[xshift=-\aluwidth/2,yshift=5*\spacing]im.north}{+}
  \node[anchor=east] (inst size) at ([xshift=-\spacing]alu incr iI) {4};
  \draw[dedge] (inst size) -- (alu incr iI);
  
  % registers
  \buildRegisterBlock{reg}{[xshift=3*\spacing]im.east}
  
  % branch alu
  \buildALU{alu branch}{[xshift=-\aluwidth/2,yshift=5*\spacing]reg.north}{+}
  
  % alu
  \buildALU{alu}{[xshift=3*\spacing+\muxwidth+\spacing] reg.east}{ALU}
  
  % immediate mux
  \node[mux,anchor=east] (immediatemux) at ([xshift=-\spacing]alu iII) {\rotatebox{90}{mux}};
  
  % data memory
  \buildDataMemoryBlock{dm}{[xshift=2*\spacing,yshift=-\lineheight]alu o}
  
  % load mux
  \node[mux,anchor=east] (loadmux) at ([xshift=4*\spacing,yshift=\spacing]reg.north east) {\rotatebox{90}{mux}};
  
  % pc mux
  \node[mux,anchor=east] (pcmux) at ([xshift=-\muxwidth,yshift=2*\spacing+\aluheight+\spacing+\muxheight/2-(\muxheight/2-\muxinputoffset)]im.north west) {\rotatebox{90}{mux}};
%  \node[mux,anchor=east] (pcmux) at ([xshift=-\muxwidth,yshift=10*\spacing]im.north west) {\rotatebox{90}{mux}};
  
  % control
  \buildControlBlock{control}{[yshift=-2*\spacing]im.south east}
  
  % control and
  \buildAND{and}{[xshift=\spacing,yshift=\spacing]$(reg.north east)!(pcmux.north)!(reg.south east)$}
  
  % control wiring
  \draw[dedge,control] (control RegWrite) -| (reg rw);
  \draw[dedge,control] (control ALUSrc) -| (immediatemux.south);
  \draw[dedge,control] (control ALUOp) -| (alu c);
  \draw[dedge,control] (alu z) |- (and iII);
  \draw[dedge,control] (and o) -| (pcmux.north);
  \draw[dedge,control] (control MemRead) -| ([xshift=-3*\spacing]dm.south);
  \draw[dedge,control] (control MemWrite) -| ([xshift=3*\spacing]dm.south);
  \draw[dedge,control] (control MemtoReg)
                    -- ([xshift=2*\spacing] $(dm.north east)!(control MemtoReg)!(dm.south east)$)
                    |- ([yshift=\spacing]loadmux.north)
                    -- (loadmux.north);
  \draw[dedge,control] (control Branch)
                    -- ([xshift=3*\spacing] $(dm.north east)!(control Branch)!(dm.south east)$)
                    |- (and iI);
  
  % wiring
  \draw[dedge] (pc out) -- (im addr);
  \draw[dedge] ($(pc out)!0.5!(im addr)$) |- (alu incr iII);
  \draw[dedge] ([xshift=-\spacing,yshift=\spacing]im.north west)
            -- ([xshift=-\spacing,yshift=\spacing]im.north east)
            |- (alu branch iI);
  \draw[dedge] (im inst)
            -- ([xshift=\spacing]im inst)
            |- ([xshift=\spacing,yshift=-\spacing]reg.south east)
            |- ([yshift=-\muxinputoffset]immediatemux.west);
  \draw[dedge] ([xshift=\spacing]im inst)
            |- (alu branch iII);
  \draw[dedge] ([xshift=-2*\spacing]reg regI)   -- (reg regI);
  \draw[dedge] ([xshift=-2*\spacing]reg regII)  -- (reg regII);
  \draw[dedge] ([xshift=-2*\spacing]reg regIII) -- (reg regIII);
  \draw[dedge] ([yshift=1*\spacing]control Instruction) -- (control Instruction);
  \draw[dedge] (reg valI) -- (alu iI);
  \draw[dedge] (reg valII) -- ([yshift=\muxinputoffset] immediatemux.west);
  \draw[dedge] ([xshift=2*\spacing]reg valII) |- (dm data);
  \draw[dedge] (immediatemux) -- (alu iII);
  \draw[dedge] (alu o) -- (dm addr);
  \draw[dedge] ([xshift=\spacing]alu o)
            |- ([yshift=-\muxinputoffset]loadmux.east);
  \draw[dedge] (dm value)
            -- ([xshift=\spacing]dm value)
            |- ([yshift=\muxinputoffset]loadmux.east);
  \draw[dedge] (loadmux.west)
            -- ([xshift=-\spacing,yshift=\spacing]reg.north west)
            |- (reg data);
  \draw[dedge] (alu incr o)
            -- ([xshift=\spacing]alu incr o)
            |- ([yshift=-\muxinputoffset]pcmux.east);
  \draw[dedge] (alu branch o)
            -- ([xshift=\spacing]alu branch o)
            |- ([yshift=\muxinputoffset]pcmux.east);
  \draw[dedge] (pcmux.west)
            -| ([xshift=-\spacing]pc.west)
            |- (pc);
  
  % connection points
  \node[point] () at ([xshift=\spacing]pc.east) {};
  \node[point] () at ([xshift=-\spacing,yshift=\spacing]im.north west) {};
  \node[point] () at ([xshift= 1*\spacing]im inst) {};
  \node[point] () at ([xshift=-2*\spacing]reg regI) {};
  \node[point] () at ([xshift=-2*\spacing]reg regII) {};
  \node[point] () at ([xshift=-2*\spacing]reg regIII) {};
  \node[point] () at ([xshift=2*\spacing]reg valII) {};
  \node[point] () at ([xshift= 1*\spacing]alu o) {};
  \node[point] () at ([yshift=1*\spacing]control Instruction) {};
\end{tikzpicture}
}

  \caption{Primitive model of a RISC-V Processor.}
  \label{fig:machine:riscv}
\end{figure}

% mux

% and

% alu

\subsubsection{Data Plane}

% pc

% instruction fetch and decode

% register read

% alu operation

% memory operation

% branching

\subsubsection{Control Plane}

% the clock cycle ticks

% ALU signals

% Branch

% MemRead & MemWrite

% RegWrite & MemtoReg

\subsection{Followup}

% why languages tend to look the same
You will find that many of the design choices of \csharp, and any other programming language for that matter, have their roots in this machine. There is a reason why these languages look like the do. At the end of the day, they have to be executed on a physical processor, and often there is really only one way of doing something on such a processor that doesn't incur a huge performance penalty.

