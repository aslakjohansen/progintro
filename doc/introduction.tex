\chapter{Introduction}
\label{sec:intro}

Hello

\section{Intended Audience}

\section{Tenets}

\begin{enumerate}
  \item The role of a course introducing programming is to create a technical foundation, not to please the industry. Ideally, students of such a course can -- in time -- bring positive change to the industry.
  \item Practice should be rooted in theory.
  \item The why is just as important as the how.
  \item Concepts are more important than concrete implementations.
  \item Concrete implementation choices can often be highlighted by contrasting to other implementations.
  \item Students should be able to look to educational material for good examples of how to write.
\end{enumerate}

\section{\idx{Reading Guide}{Reading guide}}

\subsection{Mindset}

\begin{inspiration}{\idx{Ralph Waldo Emerson}{Emerson, Ralph Waldo}\cite{selfrel}}
  \quoted{Its not the Destination, It's the journey.}
\end{inspiration}

% general: how we learn
People rarely decide to undertake the hardships of learning something because they want to go through the process of learning. They want have been through the process and know. It's the destination that pulls them in. As intelligent beings we subconsciously seek shortcuts, yet we are not particularly good at keeping the long-term goalpost present in mind. In order to absorb and understand the knowledge we need to be present along the way, observe the details and experience how they relate to each other. This applies to programming, perhaps more so than to most other fields.

% specific: this book positioned in your journey
This book is intended to help you in during the very first leg of your journey. As that is the first of many, focus will be on learning the basics rather than learning concrete tools and technologies that are used in the \idx{industry}{Industry}. At times these goals overlap, but more often than not they conflict. So, this book will not prepare you for the industry. That is a task you will have to deal with later on in your journey.

% specific: within this book
This leg of your journey can be seen as a journey in and of itself. It aims to bring you from some point to a destination. The conception that reading the book and finding the right answer to the exercises will bring you to that destination is a fallacy. There are many steps on this path, and how you tread them matter.
Copying someone else's \idx{homework}{Homework} (be it a human or an LLM) will help you \quoted{solve} the exercises, but doing so does not make you progress in your journey. These are \idx{empty calories}{Calories!Empty} and consistently going for that option is detrimental to your education. To benefit from from solving the exercises you need to go through the thought process of deriving a solution. You have to struggle a bit for your mindset and understanding to be adjusted. And for that reason, we purposefully don't take the straight path.

% sources of information in general
That is also the case when it comes to the way you work with this book (and other resources, for that matter). Information will rarely be available in your preferred format. You need to build up skills for dealing with information in formats that you have no control over, and figure out how to best incorporate that in your workflow. For instance, if the information you need is in PDF format, then you can either print it, or spend a portion of your screen real estate displaying the file. In the latter case you should choose a decent PDF viewer (e.g., not something that came with your browser), and consider using the remainder of your screen for your coding. Operating two full-screen applications will impose the penalty of a large number of \idx{context switches}{Context switch} on you. Conversely, when you are communicating to others, you will rarely be in a situation where your preferred means of communication fits that of your audience.

% TODO: Why is this not a website? no effort links does not force the reader to think about the matter at hand

% your help needed
So, we need your help in achieving this. You do so by keeping a focus on the following across your studies:
\begin{enumerate}
  \descitem{Curiosity} 
  \descitem{Pride} Programming is a craft, and it needs to be honed. One needs to build up an opinion about what it means for software to be good. This is done by repeatedly attempt to improve the state of the codebase. Handing over a codebase to someone else should be a satisfying moment; one that makes you feel proud of what you have accomplished.
  \descitem{Thoroughness} 
  \descitem{Effort} 
  \descitem{Communication} 
\end{enumerate}

\subsection{Index}

Items are grouped so that in order to look up a \say{directed edge}, one first looks up \say{edge}, and then \say{directed} among the related entries. Page numbers with definitions are marked with bold.

\section{Acknowledgments}
