\chapter{Introduction}
\label{sec:intro}

% what is programming? see a computer as a general-purpose machine, the art of making that machine take a specific form that solves a concrete problem, it turns out that there are common techniques patterns and constructs, some are simple and others are very complex, having complexity noes not imply that something is better
Before we can program we need a conceptual model of this computer thingy that we want to program. A useful way of looking at this is to see the \idx{computer}{Computer} as a general-purpose machine. The task of programming it then becomes the \idx{art}{Art} of making that machine take a specific form that allows it to solve a concrete problem or class of problems. It turns out that there are common techniques, \idx{patterns}{Pattern} and constructs that can help us out. Some are simple and others are very complex. \idx{Complexity}{Complexity}, in and of itself, is a bad thing. It makes it hard for humans (such as programmers) to grasp what is going on. But sometimes no simple solution exists, and we may be forced to accept a complex one. It is up to the programmer to reason about this, and make informed decisions of \textsl{how} the software is constructed.

% thinking about programming: somewhere between a craft and an art that is rooted in theory, we are building complex setups on a complex foundation, understanding that foundation (theory) is fundamental to constructing good software, for the vast majority of students practice is what allows them to connect with those fundamentals and begin to reason about programming
Depending on who you ask, the task of programming lands somewhere between a \idx{craft}{Craft} and an \idx{art}{Art}. No matter where on the spectrum you see it, the decisions you make while programming should be rooted in \idx{theory}{Theory}. But why, you may ask? When programming something nontrivial, we are producing complex setups on top of a complex foundation. That \idx{foundation}{Foundation} covers the hardware of the computer itself, the \idx{operating system}{Operating system} running on top of the computer, other software running on the operating system, the programming language(s) at play, libraries of existing software, and \ldots\ perhaps \ldots\ other computers over a \idx{network}{Network}. Understanding (the theory behind) that foundation is fundamental to constructing good software. Yet, for the vast majority of new \idx{students}{Student}, practice is what allows them to connect with those fundamentals and begin to reason about programming.

% high abstraction as low-hanging fruit: closest to where the value becomes obvious, for many tasks computational efficiency is not critical (processing power is cheap), yet there can be good reasons (user experience, cost of execution, ethics regarding climate change)
Some languages are considered low-level. They are suited for working with concepts that relates to the computer itself. Your operating system, or at least its kernel, is written in such a language (or languages). Other language are considered high-level. They are suited for working with concepts that relates to domains outside of the computer. This could be a card game. In reality there is a spectrum between low and high level, and there is a myriad of languages that can be placed on this spectrum. Often, and especially for non-programmers, it is easier to spot the value generated close to the \idx{problem domains}{Domain!Problem}, and these areas then to lend themselves to high-level languages. These typically make programmers more efficient while being less efficient computationally. That is, they trade \idx{execution efficiency}{Efficiency!Execution} of \idx{development efficiency}{Efficiency!Development}. As processing power is cheap and many tasks are tolerant in terms of \idx{response time}{Response time}, often computational efficiency is not considered critical or even important. Yet, there can be good reasons to care, namely (i) to improve \idx{user experience}{User experience}, (ii) to lower the cost of operations, and (iii) due to the \idx{ethics}{Ethics} of how \idx{power consumption}{Power consumption} is driving \idx{climate change}{Climate change}.

% picking high-level:
As this is an introductory course, we will mostly not care about efficiency. This textbook was written primarily with a software education in mind that is mostly focused on how software is constructed for organizations by essentially gluing existing solutions together. I would call this a \idx{high-level software education}{Software education!High-level}. In such there is little time to learn about programming, so it is important that all the programming courses are directly applicable. For this reason we focus on a high-level language, namely \csharp. This choice, as any other, comes with downsides. So, lets take a look at the main ones:
\begin{itemize}
  \item Once you have learned a language it is easy to go up in abstraction but hard to go down. When you go up in abstraction, there is somethings that you no longer have to worry about. When you go down in abstractions, something that you have come to rely on is taken away. By going the way of a high-level language in the beginning, makes it likely that you will stay there.
  \item Any high-level language will have high-level constructs, and why they may at times be easy to use, they tend to be hard to understand. It is simply far away form how the computer works. Unfortunately, that understanding is key to becoming a good programmer. A more ideal approach would be to start with a low-level language, then teach a medium-level language and finally end up with a high-level language. But this all takes time, and on a high-level software education that is unfortunately at a premium.
\end{itemize}

\section{Intended Audience}

% background: no programming experience, little understanding of computers, good understanding of internet
This book will make no assumption of previous programming experience. It will, however, expect you to have some minimal understanding of your operating system of choice. You should, for instance, know how to install software, and preferably have an understanding of how your filesystem is structured. Section \ref{bg:fs} covers what you need to know about your filesystem, but it won't make up for lack of practical experience. Finally, we expect you to have basic skills in looking up information on the internet and, critically, assessing its relevance and quality.

\section{Tenets}

\begin{enumerate}
  \item The role of a course introducing programming is to create a technical foundation, not to please the industry. Ideally, students of such a course can -- in time -- bring positive change to the industry.
  \item Practice should be rooted in theory.
  \item The why is just as important as the how, if not more so.
  \item Concepts are more important than concrete implementations.
  \item Concrete implementation choices can often be highlighted by contrasting to other implementations.
  \item Students should be able to look to educational material for good examples of how to write.
\end{enumerate}

\section{Reading Guide}

% reading order, dependencies
This book is, as so many before it, intended to be \idx{read}{Reading guide} from beginning to end. The \csharp\ programming language has a lot of intricacies that make it virtually impossible to serialize the subject into a sequence of elements whose understanding only relies on the previous elements. That is a \idx{dependency graph}{Graph!Dependency} all edges go backward in time\footnote{This should make sense after reading section \ref{sec:bg:graph}. That forward dependency should illustrate why they are bad.}.

% high-level overview
This book is organized in topics which are grouped in themes. After this introduction you will find a background chapter with a summary of the theoretical that you are expected to have. Then comes a prereqs chapter that takes you though the process of installing the software that allows you to try out the examples and solve the exercises. Then follows four parts covering relevant topics and finally you will find a bibliography and an index.

% per-topic structure (theory, practice in various languages)
Each part is made up of a number of topics. These topics are typically presented through a motivation, some generic theory, language-specific variations and finally exercises. While \csharp\ will be the natural centerpiece of the these variations, it will be contrasted to a number of other languages for context. The parts are introduced by a single example that gives a good impression of what will be covered.

\subsection{Mindset}

\begin{inspiration}{\idx{Ralph Waldo Emerson}{Emerson, Ralph Waldo}\cite{selfrel}}
  \quoted{It's not the Destination, It's the journey.}
\end{inspiration}

% general: how we learn
People rarely decide to undertake the hardships of learning something because they want to go through the process of learning. They want have been through the process and know. It's the destination that pulls them in. As intelligent beings we subconsciously seek shortcuts, yet we are not particularly good at keeping the long-term goalpost present in mind. In order to absorb and understand the knowledge we need to be present along the way, observe the details and experience how they relate to each other. This applies to programming, perhaps more so than to most other fields.

% specific: this book positioned in your journey
This book is intended to help you in during the very first leg of your journey. As that is the first of many, focus will be on learning the basics rather than learning concrete tools and technologies that are used in the \idx{industry}{Industry}. At times these goals overlap, but more often than not they conflict. So, this book will not prepare you for the industry. That is a task you will have to deal with later on in your journey.

% specific: within this book
This leg of your journey can be seen as a journey in and of itself. It aims to bring you from some point to a destination. The conception that reading the book and finding the right answer to the exercises will bring you to that destination is a fallacy. There are many steps on this path, and how you tread them matter.
Copying someone else's \idx{homework}{Homework} (be it a human or a generative AI) will help you \quoted{solve} the exercises, but doing so does not make you progress in your journey. These are \idx{empty calories}{Calories!Empty} and consistently going for that option is detrimental to your education. To benefit from from solving the exercises you need to go through the thought process of deriving a solution. You have to struggle a bit for your mindset and understanding to be adjusted. And for that reason, we purposefully don't take the straight path.

% sources of information in general
That is also the case when it comes to the way you work with this book (and other resources, for that matter). Information will rarely be available in your preferred format. You need to build up skills for dealing with information in formats that you have no control over, and figure out how to best incorporate that in your workflow. For instance, if the information you need is in PDF format, then you can either print it, or spend a portion of your screen real estate displaying the file. In the latter case you should choose a decent PDF viewer (e.g., not something that came with your browser), and consider using the remainder of your screen for your coding. Operating two full-screen applications will impose the penalty of a large number of \idx{context switches}{Context switch} on you. Conversely, when you are communicating to others, you will rarely be in a situation where your preferred means of communication fits that of your audience.

% why is this not a website? no effort links does not force the reader to think about the matter at hand
This is a significant part of the reason why this is not a website. Websites can be really good for many things. They are, for instance, good for quickly looking up definitions that it's not worth it to memorize. But that is a different problem. Learning to program requires you to interact with the matter, to tie individual parts of the material to personal experiences; be it the success of completing a small game in an exercise or being annoyed by not having everything \say{just work} and having to manually look though the index to find what you are looking for. Simply clicking through the material will not force you to involve yourself in it, and then your \idx{brain}{Brain} wont build the connections necessary for the constructed knowledge to \idx{persist}{Persistance} over time. The effort will be lost.

% your help needed
So, we need your help in achieving this. You do so by keeping a focus on the following across your studies:
\begin{enumerate}
  \descitem{Curiosity} The most important characteristic of a student of programming is that of \idx{curiosity}{Curiosity}; the desire to understand how things work and the willingness to invest time into exploring what makes them tick.
  \descitem{Pride} Programming is a craft, and it needs to be honed. One needs to build up an opinion about what it means for code to be good. This is done by repeatedly attempt to improve the state of the codebase. Handing over a codebase to someone else should be a satisfying moment; one that makes you feel proud of what you have accomplished.
  \descitem{Thoroughness} There is more to writing code than simply implementing functionality. How that functionality is achieved matter, and the quality characteristics involved are both many and intertwined through, e.g., through \idx{tradeoffs}{Tradeoff}. When learning the trade, the value of implementing the functionality is negligible, and the details are much more important.
  \descitem{Communication} Programming, like any other even remotely complex field, introduces concepts and notions. A technical \idx{vocabulary}{Vocabulary} is needed to efficiently and unambiguously reference these. It is critical to acquire this vocabulary and exercise it in your daily interactions. Then trying to convey technical information, if you don't use the appropriate technical terms, the receiving party will more likely than not misunderstand you.
  \descitem{Effort} Knowing how to best apply your time is key. Have pet projects, and choose them in such a way that they inspire your academic curiosity and/or are investments into your own ability to take in knowledge. Avoid squandering the time you put in trying to achieve arbitrary goals such as a perfect score.
\end{enumerate}

\subsection{Index}

At this point, you may have noticed some highlighted seemingly random words or phrases. This is by no accident. These are the words that appear in the index. If they, in addition to color, are underlined, then it will be a definition of a term. Entries of the index are grouped so that in order to look up a \say{directed edge}, one first looks up \say{edge}, and then \say{directed} among the related subentries. Page numbers with definitions are marked with bold.

\section{Computer Literacy}

Lets briefly take a look at how different people use computers. Some simply don't. Others do, but in reality all they use it for is as a window towards web services. In that respect, they are more users of web services that of computers. Others yet actually use their computer itself and have the fundamental understanding of it to do so. To construct software for the computer requires a deeper understanding though, and a culture. This could be summarized as:

\begin{itemize}
  \descitem{Level 0 \say{Computer Illiterate}:} Incapable of using a computer.
  \descitem{Level 1 \say{Browser Literate}:} Capable of booting a computer, starting a browser and operating the internet through it. This implies a basic understanding of what a website is, internet security, accounts, search technique and copy paste.
  \descitem{Level 2 \say{Desktop Literate}:} Capable of operating a general purpose operating system. That covers installing programs, managing local files, and understanding the file system hierarchy as well as the decoupling of file types and applications.
  \descitem{Level 3 \say{Developer Literate}:} Capable of operating a general purpose operating system as a software developer. This includes having an understanding of environment variables, standard streams, the terminal, color representation, file format representation, the processor and the memory hierarchy, and programming language syntax.
\end{itemize}

Besides teaching a programming language, \csharp, this book also aim to raise the readers computer literacy to level 3. You are expected to be somewhere between levels 1 and 3. If you are at level 0 then this book is not for you.

\section{Acknowledgments}
