\chapter{The First Program}
\label{sec:first}

\begin{inspiration}{The Linux Kernel Module Programming Guide\cite{lkmpg20070518}}
  \quoted{When the first caveman programmer chiseled the first program on the walls of the first cave computer, it was a program to paint the string `Hello, world' in Antelope pictures. Roman programming textbooks began with the `Salut, Mundi' program. I don't know what happens to people who break with this tradition, but I think it's safer not to find out.}
\end{inspiration}

\section{Phases}

% ahead-of-time compiled language, compilation phase results in a binary that is transferable (and portable) and can be executed without the precense of the compiler
\csharp\ is an ahead-of-time compiled language. This means that program execution is a step separate from program compilation. Or, to skip the technical terms, there is a compilation phase whereby a special program known as a compiler produces a program binary by processing the program source code. This binary is transferable and (fairly) portable. This means that it can be moved to a different machine and executed without the presence of a compiler. But lets take a step back a take a look of all the involved steps.

\subsection{Project Creation}

First, lets create a directory that can host our project:

\begin{minted}[]{shell-session}
aslak@gaia:/tmp$ mkdir new_project
\end{minted}

Then, we can go to that directory and initalize a new console project:

\begin{minted}[]{shell-session}
aslak@gaia:/tmp$ mkdir new_project
aslak@gaia:/tmp$ cd new_project
aslak@gaia:/tmp/new_project$ dotnet new console
The template "Console App" was created successfully.

Processing post-creation actions...
Restoring /tmp/new_project/new_project.csproj:
  Determining projects to restore...
  Restored /tmp/new_project/new_project.csproj (in 146 ms).
Restore succeeded.
\end{minted}

So, what happened? Lets take a look at what happened in our newly created directory:

\begin{minted}[]{shell-session}
aslak@gaia:/tmp/new_project$ ls
new_project.csproj  obj  Program.cs
\end{minted}

Three files were created, and one of them is a directory:
\begin{itemize}
  \item \filename{new\_project.csproj}: This is a \say{project} file. It holds \idx{metadata}{Metadata} that tells the \commandname{dotnet} command how to build the project. The build process is the process that converts our human-readable source code to a binary that can be executed on a machine.
  \item \filename{obj}: For now, we don't care much about this directory.
  \item \filename{Program.cs}: This is our program file. It is where we write our code. Later on we will add more files. From the perspective of the \commandname{dotnet} command, the name doesn't matter. However, as your project grows, humans will find is very confusing if certain standards are not met. We will cover these throughout the book.
\end{itemize}

\subsection{Writing}
\label{first:writing}

Open the \filename{Program.cs} in your text editor and make sure the contents is as follows:

\includeCsharpFile{first/hello/Program.cs}

\subsection{Saving}

Save the file. This persists to data that was present in your computers memory under the control of your text editor to disk. That way another program can easily acces it through a valid path to the file.

\subsection{Compilation}

One such program is a compiler, and that happens to be what we need to build the program binary that is needed for execution. That compiler can be called using the \commandname{dotnet} command:

\begin{minted}[]{shell-session}
aslak@gaia:/tmp/new_project$ dotnet build
  Determining projects to restore...
  All projects are up-to-date for restore.
  new_project -> /tmp/new_project/bin/Debug/net8.0/new_project.dll

Build succeeded.
    0 Warning(s)
    0 Error(s)

Time Elapsed 00:00:05.36
\end{minted}

If we list the contents of the project directory we will see that a new \filename{bin} directory has been created for storing the binary program:

\begin{minted}[]{shell-session}
aslak@gaia:/tmp/new_project$ ls
bin  new_project.csproj  obj  Program.cs
\end{minted}

\subsection{Execution}

While we say that the compiler has produced an executable binary for us, it is not a binary that our hardware is capable of running. We need a \idx{virtual machine}{Virtual machine} for that. This is essentially a special program that simulates a different architecture. The binary we have contains \idx{Intermediate Language}{Intermediate language (IL)} (IL) bytecode, and we need a \idx{Common Language Runtime}{Common Language Runtime (CLR)} (CLR) virtual machine to execute it. Luckely, once again, the \commandname{dotnet} command gives us access to such:

\begin{minted}[]{shell-session}
aslak@gaia:/tmp/new_project$ dotnet run
Hello, World!
\end{minted}

\subsection{Big Picture}

A typical development cycle looks like figure \ref{fig:first:phases:cycle}: First a project is created, then the developer writes some code that is then compiled and executed. When writing the text editor may provide hints of errors that the developer then takes into account. The \idx{compiler}{Compiler} may give provide feedback in the form of \idx{errors}{Error} or \idx{warnings}{Warning} that need to be addressed. When executing, one might discover \idx{bugs}{Bug} that needs fixing. All of this brings us back to the writing phase.

On top of that, developers usually work on a small portion of the intended functionality. When that functionality is deemed correct -- as there are no indicators of problems in our \textsl{Write}, \textsl{Compile} and \textsl{Execute} phases -- the developer takes on an other portion and starts in the \textsl{Write} phase.

\begin{figure}[tbp]
  \begin{center}
  \begin{tikzpicture}[remember picture]
    \newcommand{\stepsize}{24mm}
    \tikzstyle{edge}  = [thick,>=stealth,draw=black]
    \tikzstyle{dedge} = [thick,->,>=stealth,draw=black]
    \tikzstyle{node}=[
      circle,
      draw=purple,
      anchor=center,
      thick,
      minimum size=16mm,
    ]
    
    \node[node] (nCreate) at (0*\stepsize,0) {Create};
    \node[node] (nWrite) at (1*\stepsize,0) {Write};
    \node[node] (nSave) at (2*\stepsize,0) {Save};
    \node[node] (nCompile) at (3*\stepsize,0) {Compile};
    \node[node] (nExecute) at (4*\stepsize,0) {Execute};
    
    \draw[dedge] (nCreate)--(nWrite);
    \draw[dedge] (nWrite)--(nSave);
    \draw[dedge] (nSave)--(nCompile);
    \draw[dedge] (nCompile)--(nExecute);
    \draw[dedge] (nCompile)--([yshift= 6mm]nCompile.north)-|(nWrite);
    \draw[dedge] (nExecute)--([yshift=-6mm]nExecute.south)-|(nWrite);
  \end{tikzpicture}
\end{center}

  \caption{Typical developer cycle.}
  \label{fig:first:phases:cycle}
\end{figure}

\csharpsection{\csharp}

% intro
The program from section \ref{first:writing} looks trivial on the surface, but there happens to be quite a bit more to it that deserves some attention. So, lets give in to it! After all, there are fundamental details at play here that is going to inform our understanding of the \csharp\ programming language.

% compilation
We will start by looking into how \csharp\ code is \idx{compiled}{Compilation} into an \idx{executable}{Executable} binary. This happens in four distinct steps as illustrated in figure \ref{fig:first:csharp:compilation:phases}. The following subsections will cover the role of each of these phases, and give a very short introduction to the mechanisms behind them.

\begin{figure}[tbp]
  \begin{center}
  \scalebox{0.885}{
    \begin{tikzpicture}[remember picture]
      \newcommand{\spacing}[0]{ 10mm }
      \newcommand{\filesize}[0]{ 0.5 }
      \tikzstyle{edge}  = [thick,>=stealth,draw=black]
      \tikzstyle{dedge} = [thick,->,>=stealth,draw=black]
      \tikzstyle{node}=[
        circle,
        draw=purple!70,
        fill=purple!10,
        anchor=west,
        align=center,
        very thick,
        minimum size=1.85cm
      ]
      \tikzstyle{label}=[
        anchor=north,
        align=center,
      ]
      
      \tikzset{
        pics/file/.style args={#1}{
          code={
            \newcommand{\fileh}[0]{1.0}
            \newcommand{\filew}[0]{0.8}
            \newcommand{\filehc}[0]{\fileh/2}
            \newcommand{\filewc}[0]{\filew/2}
            \draw[-,thick]
              (-\filew *#1, \fileh *#1) to % north west
              ( \filewc*#1, \fileh *#1) to % north east 1
              ( \filew *#1, \filehc*#1) to % north east 2
              ( \filew *#1,-\fileh *#1) to % south east
              (-\filew *#1,-\fileh *#1) to % south west
              (-\filew *#1, \fileh *#1)    % north west
              ;
            \draw[-,thick]
              ( \filewc*#1, \fileh *#1) to
              ( \filewc*#1, \filehc*#1) to
              ( \filew *#1, \filehc*#1)
              ;
          }
        },
      }
      
      % ordinary nodes
      
      \node[node] (lexer)     at (0,0) {Lexer};
      \node[node] (parser)    at ([xshift=2*\spacing]lexer.east) {Parser};
      \node[node] (checker)   at ([xshift=2*\spacing]parser.east) {Type\\Checker};
      \node[node] (generator) at ([xshift=2*\spacing]checker.east) {Generator};
      
      % file nodes
      
      \draw[draw=purple] ([xshift=-\spacing]lexer.west) pic{file={\filesize}};
      \node[label] () at ([xshift=-\spacing,yshift=-2*\filesize*1cm]lexer.west) {\csharp\\file};
      
      \draw[draw=purple] ([xshift=\spacing]lexer.east) pic{file={\filesize}};
      \node[label] () at ([xshift=\spacing,yshift=-2*\filesize*1cm]lexer.east) {tokens};
      
      \draw[draw=purple] ([xshift=\spacing]parser.east) pic{file={\filesize}};
      \node[label] () at ([xshift=\spacing,yshift=-2*\filesize*1cm]parser.east) {parse\\tree};
      
      \draw[draw=purple] ([xshift=\spacing]checker.east) pic{file={\filesize}};
      \node[label] () at ([xshift=\spacing,yshift=-2*\filesize*1cm]checker.east) {parse\\tree};
      
      \draw[draw=purple] ([xshift=\spacing]generator.east) pic{file={\filesize}};
      \node[label] () at ([xshift=\spacing,yshift=-2*\filesize*1cm]generator.east) {binary};
      
      % edges
      
      \draw[dedge] ([xshift=-6mm]lexer.west)--(lexer);
      \draw[dedge] (lexer)--([xshift=6mm]lexer.east);
      
      \draw[dedge] ([xshift=-6mm]parser.west)--(parser);
      \draw[dedge] (parser)--([xshift=6mm]parser.east);
      
      \draw[dedge] ([xshift=-6mm]checker.west)--(checker);
      \draw[dedge] (checker)--([xshift=6mm]checker.east);
      
      \draw[dedge] ([xshift=-6mm]generator.west)--(generator);
      \draw[dedge] (generator)--([xshift=6mm]generator.east);
    \end{tikzpicture}
  }
\end{center}

  \caption{Compilation phases of a \csharp\ program.}
  \label{fig:first:csharp:compilation:phases}
\end{figure}

\subsection{Lexing}

% tokens, token type annotation
The first thing a compiler will do when it sees some code, is to hand it to its \idx{lexer}{Lexer} that then chop it up into a sequence of \idx{tokens}{Token}. This is the only job of the lexer. Each token represents a small part of the code. These parts are atomic units with respect to the following phases of the compilation. By this, we mean that there is no reason for any of the following phases to deal with the code at a higher granularity than this. This may seem very abstract at this point, and that's okay. Lets just say that the output of the lexer is as illustrated in figure \ref{fig:first:hello:tokens}. In this figure, each token is annotated by \idx{type}{Type}. Some are simply named after the character they represent. These include \texttt{RPAR} for right parenthesis and \texttt{SCOLON} for semicolon. Others represent a sequence of characters.

\begin{figure}[tbp]
  \begin{center}
  \begin{tikzpicture}[]
    \newcommand{\spacing}[0]{ 6mm }
    \tikzstyle{dedge} = [thick,->,>=stealth,draw=black]
    \tikzstyle{shared}=[
      anchor=base east,
    ]
    \tikzstyle{structure}=[
      shared,
    ]
    \tikzstyle{token}=[
      shared,
      text=teal,
      font=\ttfamily,
    ]
    \tikzstyle{label}=[
      font=\ttfamily,
    ]
    
    \node[matrix,column sep=0mm,anchor=north,] () at (0,0) {
      \node[structure] () {[};
      &
      \node[token] (console) {Console};
      &
      \node[structure] () {,};
      &
      \node[token] (period) {.};
      &
      \node[structure] () {,};
      &
      \node[token] (writeline) {WriteLine};
      &
      \node[structure] () {,};
      &
      \node[token] (lpar) {(};
      &
      \node[structure] () {,};
      &
      \node[token] (string) {"Hello, World!"};
      &
      \node[structure] () {,};
      &
      \node[token] (rpar) {)};
      &
      \node[structure] () {,};
      &
      \node[token] (scolon) {;};
      &
      \node[structure] () {]};
      \\
    };
    
    \node[label] (console label) at ([yshift=\spacing]console.north) {NAME};
    \node[label] (period label) at ([yshift=-\spacing]period.south) {PERIOD};
    \node[label] (writeline label) at ([yshift=\spacing]writeline.north) {NAME};
    \node[label] (lpar label) at ([yshift=-\spacing]lpar.south) {LPAR};
    \node[label] (string label) at ([yshift=\spacing]string.north) {STRING};
    \node[label] (rpar label) at ([yshift=-\spacing]rpar.south) {RPAR};
    \node[label] (scolon label) at ([yshift=\spacing]scolon.north) {SCOLON};
    
    \draw[dedge] (console label)--(console);
    \draw[dedge] (period label)--(period);
    \draw[dedge] (writeline label)--(writeline);
    \draw[dedge] (lpar label)--(lpar);
    \draw[dedge] (string label)--(string);
    \draw[dedge] (rpar label)--(rpar);
    \draw[dedge] (scolon label)--(scolon);
  \end{tikzpicture}
\end{center}

  \caption{Token sequence of the hello-world program produced by the lexer.}
  \label{fig:first:hello:tokens}
\end{figure}

% the name token
\texttt{NAME}, for instance, is our first encounter with the concept of \idx{naming}{Naming}. We will dig more into this as we progress through this book. But for now, lets just say that it is a name which can be used to reference \textsl{something}. Such names are defined by programmers and referenced by programmers having a shared \idx{context}. In \csharp\ a \texttt{NAME} has to start with either an underscore character or any alphabetic character, and the remaining characters have to be underscores, alphabetic characters or decimal digits.

% the literal token
Another example is \texttt{STRING} (also called a \idx{string literal}{Literal!String}). This is a sequence of arbitrary characters surrounded by quotes. This is a gross simplification. But for now, we don't have to worry about the details. Those worries can wait until section \ref{sec:strings}

\subsection{Grammars}

% introduction
When you read this book, we rely on a shared understanding of the syntax and semantics of the English language. Throughout the book we extend our shared vocabulary with terminology relevant for this particular domain, just like any other textbook. Every language intended for communication between humans has such syntactic rules and a vocabulary. Likewise, every programming language has a \idx{grammar}{Grammar}. It defines what is syntactically valid code.

% dealing with incorrectness
When a written \idx{English}{English} message is used to carry information between humans, the receiving party has the ability to reason about flaws in the \textsl{coding} to that message. When I miss a comma, you will still get the message. A typo is not the end of the world. If I use a word you don't know, then you will look it up, or walk away with perhaps 90\% of the meaning of the offending paragraph. As humans, we are quite robust in terms interpreting text which deviates from the expected format. This is not the case for compilers. When a compiler receives code that does not fit the grammar of the programming language it expects it will outright reject it, and by doing so force you to correct it.

% dumb as a feature
The reason for this is quite simple: While one could make the compiler attempt to guess your intention, this would represent an ambiguity that can have unforeseen consequences. For instance, it could easily change the result of a computation. In section \ref{sec:flow:branch:danglingelse}, we will cover a good example of this. The kind of mess-ups that we see from \idx{generative AI}{AI!Generative} today are simply not acceptable in most production systems. As generative AI dumps the price of low-quality code (e.g., through \idx{vibe coding}{Vibe coding}), we may start to see a shift in peoples mindset and a lowering of expectations. But for now and the foreseeable future, the compiler being dumb should very much be considered a feature.

% specific grammar of hello world example: intro
But lets get back to the hello world example. The specific grammatical rules relevant for understanding how a compiler sees the code are illustrated in syntax \ref{syntax:first:hello} as a railroad diagram. Later on, you might encounter the same information in a textual form called \idx{Backus-Naur Form}{Backus-Naur Form} (or simply BNF). While that format is more relevant for programmatic use, in this book we stick to the visually more appealing \defi{railroad diagram}{Diagram!Railroad}.

% TODO: remove name and shift it to the lexer section
\begin{syntaxfloat}
  \begin{syntax}[[xshift=24mm]concept.west]{pgm}
  \SyntaxWestSplit{MainWest}
  \SyntaxEastSplit{MainEast}
  
  \node[nonterminal] (ruleIa) at ($(begin)!0.5!(end)$) {stmts};
  
  \draw[path] (begin)--(ruleIa)--(end);
\end{syntax}
\vspace{-2mm}

\begin{syntax}[[xshift=24mm]concept.west]{stmts}
  \SyntaxWestSplit{MainWest}
  \SyntaxEastSplit{MainEast}
  
  \node[sequence] () at ([yshift=-0*\syntaxruledist]$(begin)!0.5!(end)$) {
    \node[nonterminal]    (ruleIa) {stmt};
    \\
  };
  
  \draw[path] (begin)--(ruleIa)--(end);
  \draw[path] (ruleIa)
            -|([xshift= 0.5*\syntaxruledist,yshift=0.4*\syntaxruledist]ruleIa.east)
            |-(                            [yshift=0.8*\syntaxruledist]ruleIa.center)
            -|([xshift=-0.5*\syntaxruledist,yshift=0.4*\syntaxruledist]ruleIa.west)
            |-(ruleIa);
\end{syntax}
\begin{syntax}[[xshift=24mm]concept.west]{stmt}
  \SyntaxWestSplit{MainWest}
  \SyntaxEastSplit{MainEast}
  
  \node[sequence] () at ([yshift=-0*\syntaxruledist]$(begin)!0.5!(end)$) {
    \node[nonterminal] (ruleIa) {name};
    &
    \node[terminal]    (ruleIb) {.};
    &
    \node[nonterminal] (ruleIc) {name};
    &
    \node[terminal]    (ruleId) {(};
    &
    \node[nonterminal] (ruleIe) {param-list};
    &
    \node[terminal]    (ruleIf) {)};
    &
    \node[terminal]    (ruleIg) {;};
    \\
  };
  
  \draw[path] (begin)--(ruleIa)--(ruleIb)--(ruleIc)--(ruleId)--(ruleIe)--(ruleIf)--(ruleIg)--(end);
\end{syntax}
\begin{syntax}[[xshift=24mm]concept.west]{param-list}
  \SyntaxWestSplit{MainWest}
  \SyntaxEastSplit{MainEast}
  
  \node[sequence] () at ([yshift=-1*\syntaxruledist]$(begin)!0.5!(end)$) {
    \node[nonterminal] (ruleIa) {expr};
    \\
  };
  \node[sequence] () at ([yshift=-2*\syntaxruledist]$(begin)!0.5!(end)$) {
    \node[terminal] (ruleIIa) {,};
    \\
  };
  
  \draw[path] (begin)--(end);
  \draw[path] (begin) to[-|-] (ruleIa) to[-|-] (end);
  \draw[path] (ruleIa)
            -|([xshift= 0.5*\syntaxruledist,yshift=-0.5*\syntaxruledist]ruleIa.east)
            |-(                            [yshift=0.0*\syntaxruledist]ruleIIa.east);
  \draw[path] (ruleIIa)
            -|([xshift=-0.5*\syntaxruledist,yshift=-0.5*\syntaxruledist]ruleIa.west)
            |-(                            [yshift=0.0*\syntaxruledist]ruleIa.west);
\end{syntax}
\begin{syntax}[[xshift=24mm]concept.west]{expr}
  \SyntaxWestSplit{MainWest}
  \SyntaxEastSplit{MainEast}
  
  \node[sequence] () at ([yshift=-0*\syntaxruledist]$(begin)!0.5!(end)$) {
    \node[nonterminal]    (ruleIa) {string-lit};
    \\
  };
  
  \draw[path] (begin)--(ruleIa)--(end);
\end{syntax}
%\begin{syntax}[[xshift=24mm]concept.west]{name}
%  \SyntaxWestSplit{MainWest}
%  \SyntaxEastSplit{MainEast}
%  
%  \coordinate (head) at ($(begin)!1/3!(end)$);
%  \coordinate (tail) at ($(begin)!2/3!(end)$);
%  \coordinate (center) at ($(begin)!0.5!(end)$);
%  \coordinate (tailcenter) at ($(begin)!2.5/3!(end)$);
%  
%  \node[terminal]    (ruleIa) at ([yshift= 0.5*\syntaxruledist]head) {\_};
%  \node[nonterminal] (ruleIb) at ([yshift=-0.5*\syntaxruledist]head) {azAZ};
%  \node[terminal]    (ruleIIa) at ([yshift= 1.0*\syntaxruledist]tail) {\_};
%  \node[nonterminal] (ruleIIb) at ([yshift= 0.0*\syntaxruledist]tail) {azAZ};
%  \node[nonterminal] (ruleIIc) at ([yshift=-1.0*\syntaxruledist]tail) {09};
%  
%  % head west
%  \draw[path] (begin)
%            --($([xshift=-0.5*\syntaxruledist]ruleIb.west |- begin)$)
%            |-(ruleIa.west);
%  \draw[path] (begin)
%            --($([xshift=-0.5*\syntaxruledist]ruleIb.west |- begin)$)
%            |-(ruleIb.west);
%  
%  % head east
%  \draw[path] (ruleIa.east)
%            --($([xshift=0.5*\syntaxruledist]ruleIb.east |- ruleIa)$)
%            |-(ruleIIb.west);
%  \draw[path] (ruleIb.east)
%            --([xshift=0.5*\syntaxruledist]ruleIb.east)
%            |-(ruleIIb.west);
%  
%  % tail west
%  \draw[path] (center)
%            --([xshift=-0.5*\syntaxruledist]ruleIIb.west)
%            |-(ruleIIa);
%  \draw[path] (center)
%            --([xshift=-0.5*\syntaxruledist]ruleIIb.west)
%            |-(ruleIIc);
%  
%  % tail east
%  \draw[path] (ruleIIb.east)--(end);
%  \draw[path] (ruleIIa.east)
%            --($([xshift=0.5*\syntaxruledist]ruleIIb.east |- ruleIIa)$)
%            |-(end);
%  \draw[path] (ruleIIc.east)
%            --($([xshift=0.5*\syntaxruledist]ruleIIb.east |- ruleIIc)$)
%            |-(end);
%  
%  % tail bypass
%  \draw[path] (center)
%            --([xshift=-0.5*\syntaxruledist]ruleIIb.west)
%            |-([yshift=-1.8*\syntaxruledist]ruleIIb.center)
%            -|([xshift= 0.5*\syntaxruledist]ruleIIb.east)
%            --(end);
%  
%  % tail loop
%  \draw[path] (ruleIIb.east)
%            --(tailcenter)
%            |-([yshift= 1.8*\syntaxruledist]ruleIIb.center)
%            -|(center)
%            --(ruleIIb.west);
%  
%\end{syntax}

  \caption{Components of basic hello-world program}
  \label{syntax:first:hello}
\end{syntaxfloat}

% concepts, tracks, pgm example
This grammar defines five grammatical concepts, namely \textcolor{purple}{pgm}, \textcolor{purple}{stmt}, \textcolor{purple}{param-list}, \textcolor{purple}{expr} and \textcolor{purple}{name}. Each of these definitions define the valid sequences of tokens from an empty circle on the left to a filled circle on the right. We read these by following the tracks. The program itself (or \textcolor{purple}{pgm}) thus has to contain a sequence of at least one statement (or \textcolor{purple}{stmt}). This, due to the loop around \textcolor{purple}{stmt}. % TODO: chose a color for the concepts that doesn't clash with index entries

% stmt concept, param-list concept
The next rule unfolds what a statement is. A statement must consist of a name followed by a period followed by another name followed by a left parenthesis followed by a parameter list (or \textcolor{purple}{param-list}) followed by a right parenthesis followed by a semicolon. The next rule states that this \textcolor{purple}{param-list} is either empty, or a sequence of expressions (\textcolor{purple}{expr}) with commas in between. The \textcolor{purple}{expr} has to be a string literal (\textcolor{purple}{string-lit}).

% terminals and nonterminals
By now, you should have noticed the two different types of nodes in the diagram. One type, the \idx{terminals}{Terminal!Grammar context}, represents a single token like a comma or a left parenthesis. The other type, the \idx{nonterminals}{Nonterminal}, represents another concept that can be unfolded or expanded. So, why those names in particular? This has to do with wether the node in question \textsl{terminates} the unfolding of concept nodes: While a program is a complex thing that can take many forms and needs further detailing to be fully described, a comma is simply a comma.

\subsection{Parsing}

% parse tree
That sequence of tokens is fed into the \idx{parser}{Parser}. Informed by the grammar of the language, the parser finds a unique application of the rules that match the sequence of tokens. The result is a tree structure called a \idx{parse tree}{Parse tree}. Each unfolding of a rule results in child nodes being added to the tree. The hello world program results in the parse tree of figure \ref{fig:first:hello:parsetree}. The parse tree contains only those elements whose values could vary from rule application to rule application. So, each comma or parenthesis is left out. Should the parser fail to produce a parse tree from its input tokens, then it will \idx{exit}{Exit} with a \idx{parse error}{Error!Parse}.

\begin{figure}[tbp]
  \begin{center}
  \begin{tikzpicture}[]
    \newcommand{\spacing}[0]{ 8mm }
    \tikzstyle{edge}  = [thick,>=stealth,draw=black]
    \tikzstyle{dedge} = [thick,->,>=stealth,draw=black]
    \tikzstyle{terminal}=[
      anchor=north,
    ]
    \tikzstyle{nonterminal}=[
      circle,
      draw=purple!70,
      fill=purple!10,
      anchor=north,
      very thick,
      minimum size=2.0cm
    ]
    
    \node[nonterminal] (pgm) at (0,0) {pgm};
    \node[nonterminal] (stmts) at ([yshift=-\spacing]pgm.south) {stmts};
    \node[nonterminal] (stmt) at ([yshift=-\spacing]stmts.south) {stmt};
    \node[matrix,column sep=\spacing,anchor=north,] () at ([yshift=-\spacing]stmt.south) {
      \node[nonterminal] (context) {name};
      &
      \node[nonterminal] (function) {name};
      &
      \node[nonterminal] (params) {param-list};
      \\
    };
    \node[terminal] (context value)  at ([yshift=-\spacing]context.south) {Console};
    \node[terminal] (function value) at ([yshift=-\spacing]function.south) {WriteLine};
    \node[nonterminal] (params expr)   at ([yshift=-\spacing]params.south) {expr};
    \node[nonterminal] (string)   at ([yshift=-\spacing]params expr.south) {string-lit};
    \node[terminal] (string value)   at ([yshift=-\spacing]string.south) {\texttt{"Hello, World!"}};
    
    \draw[edge] (pgm)--(stmts);
    \draw[edge] (stmts)--(stmt);
    \draw[edge] (stmt)--(context);
    \draw[edge] (stmt)--(function);
    \draw[edge] (stmt)--(params);
    \draw[edge] (context)--(context value);
    \draw[edge] (function)--(function value);
    \draw[edge] (params)--(params expr);
    \draw[edge] (params expr)--(string);
    \draw[edge] (string)--(string value);
  \end{tikzpicture}
\end{center}

  \caption{Parse tree of hello-world program.}
  \label{fig:first:hello:parsetree}
\end{figure}

\subsection{Type Checker}

% statements, expressions, types
When the final program is executing, the parts of that program that comes from an expression will \idx{evaluate}{Evaluation} to some value. The parts that comes from a statement does not. As we will see in the next chapter, any value has a type. Many of the grammatical rules that deal with expression have restrictions on the types of those expressions. For instance, you can divide a number by another number, but not by a string (such as \texttt{"Hello, World!"}).

% type checker
The parse tree is fed into yet another component of the compiler called a \idx{type checker}{Type checker}. The type checker has access to a suite of type rules. It's job is to verify that none of them are violated. Any violation is a \idx{type error}{Error!Type}.

\subsection{Compilation}

The rules that govern the evaluation of the final program operate on the final program. But that version of the program maintains much of the structure of the parse tree.

\subsection{Importance}

% point
While it is not critical to understand these intricacies in your day-to-day practice as a programmer, they do a great job at helping you \idx{reason}{Reason} about what you experience, and that saves you a lot of both effort and frustration. The takeaway should be that the compiler goes through a number of phases. Each of these have a distinct function and if you provide the compiler with invalid input (i.e., your source code), one of them will fail.

% final words on the grammar so far
As we move through this book, and broaden our understanding of the \csharp\ language, we will extend on this grammar. The pgm definition will receive one extension and both stmt and expr will receive many. We settle for these definitions being \textsl{mostly} correct as the correct rules are at least an order of magnitude more complicated, and this complexity is disruptive for understanding of the language at this level.

\pythonsection{Python}
\label{sec:first:python}

In Python, the equivalent program looks like this:

\includePythonFile{first/hello/hello.py}

In order to execute this script, we first need to make the file \idx{executable}{File!Executable}:
\begin{minted}[]{shell-session}
aslak@gaia:/tmp/python_hello$ chmod u+x hello.py
\end{minted}

That first line of the Python source code is called a \idx{shebang}{Shebang} after the first two characters. When executing, this line uses the \filename{/bin/env} program to look up an appropriate interpreter for \filename{python3} and then feed the remaining lines to it. This is known as a \idx{dispatch mechanism}{Dispatch mechanism}. The development cycle is very similar to \csharp\ except that there is neither an initial project creation step or a separate compilation step. Running the program is done like so:

\begin{minted}[]{shell-session}
aslak@gaia:/tmp/python_hello$ ./hello.py
Hello, World!
\end{minted}

The Python interpreter has an \idx{interactive mode}{Mode!Interactive} that we can \idx{invoke}{Invoke} directly:

\begin{minted}[]{shell-session}
aslak@gaia:/tmp/python_hello$ /bin/env python3
Python 3.13.3 (main, Apr 10 2025, 21:38:51) [GCC 14.2.0] on linux
Type "help", "copyright", "credits" or "license" for more information.
>>> print("Hello, World!")
Hello, World!
>>> 
\end{minted}

On the first line we invoke the interpreter. Then, it -- exactly as our shell -- presents us with a prompt. This prompt allows us to \idx{issue}{Issue} python commands. The interactive mode is typically used for rapid experimentation. Such an interactive mode is commonly known as a \defi{REPL}{REPL} (\underline{r}equest-\underline{e}val-\underline{p}rint-\underline{l}oop). That is, a program that requests a line of input from the user, evaluates that line, prints out the result and loops back to requesting a new line.

Another way to write python code is though a \idx{notebook}{Notebook} environment. Such an environment is usually provided through a browser by a system such as \idx{Jupyter}{Jypyter}. In such an environment, code is split up into cells and these cells are laid out in a sequence. Any cell can build on the code that is introduced in previous cells and the code in these cells have easily accessible means for producing visual outputs. The workflow is essentially the same as for \csharp\ except that notebooks are usually saved automatically.

\csection{C}

The minimal hello world program in C looks (more or less) like this:

\includeCFile{first/hello/hello.c}

This version immediately looks a lot more complicated! So, why is that? Well, lets try to disect it \ldots

\subsection{Explanation}

% first line
Most \idx{general-purpose languages}{Language!General-purpose} (like \csharp, C, Python and Elixir) come with a sizeable \idx{library}{Library} of functionality. While all of this is available to you as a programmer there are downsides to having it all adirectly available all the time. To simplify grossly, it is a situation similar to finding a specific needle in a stack of slighty different needles. The first line tells C -- or rather the C preprocessor -- to include the definitions present in the \filename{stdio.h} file. This pulls in, amongst others, the \funcname{printf} function, that can print stuff to the screen for us.

% main function
Wrapping use of that \funcname{printf} function is the declaration of \funcname{main}. This is the main entry point of the program. That means that this is where the execution of the program starts. It informs the program of how many options have been passed to it (through \varname{argc}) and what those options are (through \varname{argv}). This is \idx{kernels}{Kernel} primary means of informing a program of what inputs it should operate on. The \typename{int} before \funcname{main} indicates that the program evaluates to an integer value. It is convention that a zero indicates success and a non-zero value functions an an \idx{error code}{Error code}. As this program doesn't explicitly return such a number, it will default to zero.

It is important to point out that \csharp\ supports a very similar program structure. After all, this is how the kernel passes options to a program. Any language that supports such options will have something similar. It is typically in a \idx{\funcname{main} function}{Function!Main}. But sometimes it is implicit or has to be actively \idx{imported}{Importing}. That is not likely to mean much to you at this point, and that is okay.

\subsection{Workflow}

In order to compile the code to an executable binary we need to invoke a compiler. Lets use the \commandname{clang} compiler and explicitly name the file that should hold the resulting binary:

\begin{minted}[]{shell-session}
aslak@gaia:/tmp/c_hello$ clang hello.c -o hello
\end{minted}

One key difference, compared to \csharp, is that C compiles to a \defi{native binary}{Binary!Native}. That is a binary that follows a format that is directly compatible with the processor architecture of your computer. We notice this when executing the code. With \csharp we used \commandname{dotnet} to essentially simulate a virtual machine that was capable of interpreting the intermediate language binary. This manifested as a command that began with \commandname{dotnet}.

The \commandname{dotnet} program that is then invoked is another example of a native binary. So, we can invoke the binary that we have built from our C code in exactly the same way. However, unlike \commandname{dotnet}, our binary is not in the \idx{path}{Path}. That means that we need an \idx{explicit path}{Path!Explicit} to it. That is accomplished by prefacing it with \filename{./}, like so:

\begin{minted}[]{shell-session}
aslak@gaia:/tmp/c_hello$ ./hello
Hello, World!
\end{minted}

\elixirsection{Elixir}

\subsection{Interpretation}

Elixir code can be executed through an interpreter (like Python), it can be worked on through a notebook system (like Python) and it can be executed on a virtual machine (like \csharp). This virtual machine is called \idx{BEAM}{BEAM}. Unlike that for \csharp, it is a \idx{distributed}{Virtual machine!Distributed} virtual machine. The hello world program looks slightly different depending on whether a separate compilation phase is used or the code is interpreted. For the script version, the it looks like this:

\includeElixirFile{first/hello/hello.exs}

% transition to REPL
As with the Python script (in section \ref{sec:first:python}) it uses a \idx{shebang}{Shebang} mechanism. For that mechanism to work it needs to be \idx{executable}{Executable}. However, while the interpreter for the script is called \commandname{elixir} this is not the case for the REPL. The interpreter for the REPL is called \commandname{iex}:

\begin{minted}[breaklines]{shell-session}
aslak@gaia:/tmp/elixir_hello$ iex
Erlang/OTP 27 [erts-15.2.1] [source] [64-bit] [smp:16:16] [ds:16:16:10] [async-threads:1] [jit:ns]

Interactive Elixir (1.18.2) - press Ctrl+C to exit (type h() ENTER for help)
iex(1)> IO.puts("Hello, World!")
Hello, World!
:ok
iex(2)> 
\end{minted}
The line that makes the printout appear is still the same. Here, you see it appear after a \idx{prompt}{Prompt}. The \commandname{iex} prompt is, unlike the Python one numbered. Also, everything in Elixir evaluated to a value, and the \idx{REPL}{REPL} prints out this value after having evaluated a line. In this case \texttt{IO.puts("Hello, World!")}, after having performed its function of printing \quoted{Hello, World!} to the screen, evaluates to \texttt{:ok}.

\subsection{Compilation}

% mix creating new project
Like \csharp, Elixir has a notion of a project. A new project can be started using the \commandname{mix} command like so:

\begin{minted}[breaklines]{shell-session}
aslak@gaia:/tmp/elixir$ mix new hello
* creating README.md
* creating .formatter.exs
* creating .gitignore
* creating mix.exs
* creating lib
* creating lib/hello.ex
* creating test
* creating test/test_helper.exs
* creating test/hello_test.exs

Your Mix project was created successfully.
You can use "mix" to compile it, test it, and more:

    cd hello
    mix test

Run "mix help" for more commands
\end{minted}

Lets \commandname{cd} to the newly created directory and make sure that \filename{lib/hello.ex} contains the following code:

\includeElixirFile{first/hello_compiled/lib/hello.ex}

We can compile the project using the following \commandname{mix} command:

\begin{minted}[breaklines]{shell-session}
aslak@gaia:/tmp/elixir/hello$ mix compile
Compiling 1 file (.ex)
Generated hello app
\end{minted}

The BEAM virtual machine is in some ways more complex than the \idx{CLR}{CLR}. One such way is that it can run multiple applications at the same time. In order to do so, we need to define an \idx{application}{Application} in our \idx{project}{Project} and tell it to invoke our \funcname{Hello.hello} function. This, however, is not an Elixir book, and that is a detail that does not help us progress.

% interpreter with access to new project
You should note that the project does compile. One reason for that is that we can use the compiled code through our REPL. We do this by passing the \commandname{-S mix} paramter along to our \commandname{iex} command:

\begin{minted}[breaklines]{shell-session}
aslak@gaia:/tmp/elixir/hello$ iex -S mix
Erlang/OTP 27 [erts-15.2.1] [source] [64-bit] [smp:16:16] [ds:16:16:10] [async-threads:1] [jit:ns]

Interactive Elixir (1.18.2) - press Ctrl+C to exit (type h() ENTER for help)
iex(1)> Hello.hello()
Hello, World
:ok
\end{minted}

\subsection{Notebook}

% livebook
Like Python, Elixir code can run in notebook form. Unlike Python, we will actually be using this later on. So, lets see what it looks like in \idx{Livebook}{Livebook}, the Elixir application for editing and executing notebooks.

% interface walkthrough: main page, new workbook
Once started, the Livebook program exposes a webserver that hosts a webapp. Point a browser at the URL of this webserver and you will be greated by the webapp. Depending on the specific version you should see something like the screenshot of figure \ref{fig:first:elixir:livebook:main}. At the top righthand corner there is a blue button labeled \say{+ New notebook}. Next to it you will find a button labeled \say{Open} that you can upse to open a previously saved notebook. Click the first button to create a new notebook.

\begin{figure}[tbp]
  
  \caption{Screenshot of the main page of Livebook.}
  \label{fig:first:elixir:livebook:main}
\end{figure}

% configuring livebook: ligatures, automatic close brackets
Livebook has a few settings that we would like to change in order to improve our experience. To do so, open up Livebook, go to Settings and scroll down to the \say{Code Editer} segment. Make sure that:
\begin{itemize}
  \item \textbf{\textsl{Automatically close brackets}} is disabled.
  \item \textbf{\textsl{Render ligatures}} is enabled.
\end{itemize}

\begin{figure}[tbp]
  
  \caption{Screenshot of the configuration page of Livebook.}
  \label{fig:first:elixir:livebook:config}
\end{figure}

% interface walkthrough: code cells, evaluation, automatic evaluation, status indicators, saving

\begin{figure}[tbp]
  
  \caption{Screenshot of a notebook in Livebook.}
  \label{fig:first:elixir:livebook:notebook}
\end{figure}

\subsection{Clustering}

% interpreter/livebook hooking into other virtual machine

First, we need to start a named node and provide it with a cookie (read: secret). There are a number of ways to accomplish this, but the easiest one is to start a new REPL session using the \commandname{iex} command with a few parameters:

\begin{minted}[breaklines]{shell-session}
aslak@gaia:/tmp/elixir/hello$ iex --name test@127.0.0.1 --cookie mycookie 
Erlang/OTP 27 [erts-15.2.5] [source] [64-bit] [smp:16:16] [ds:16:16:10] [async-threads:1] [jit:ns]

Interactive Elixir (1.18.3) - press Ctrl+C to exit (type h() ENTER for help)
iex(test@127.0.0.1)1>
\end{minted}

Then, create a new notebook in Livebook and add four Elixir code cells to it. In the first cell we make the node hosting the livebook connect to the node of the REPL (this only has to be accomplished once):

\begin{minted}[breaklines]{elixir}
node = :"test@127.0.0.1"
cookie = :mycookie

Node.set_cookie(node, cookie)
Node.connect(node)
Node.alive?()
\end{minted}

This should evaluate to \valuename{true}. Next, we can run some code on the node we just connected to. In this case, we wait for a message of a particular format to be received and send back an appropriate response:

\begin{minted}[breaklines]{elixir}
pid = Node.spawn_link node, fn ->
  receive do
    {:ping, org_pid} -> send org_pid, {:pong, "Hello, World"}
  end
end
\end{minted}

The third cell contains code for sending a message containing the process ID of the current process to the process that we just started on the other node. The code for this is:

\begin{minted}[breaklines]{elixir}
send pid, {:ping, self()}
\end{minted}

At this point we can use the fourth cell to wait for a message to arrive:

\begin{minted}[breaklines]{elixir}
receive do
  message -> message
end
\end{minted}

\textbf{Note:} This is a very simple example of how code can communicate across a clustered virtual machine. While this may seem overwhelming to you now, it can both get much more complicated and much more beneficial.

\section{Summary}

% ref to fig, what does that mean? Is it better to have more boxes checked? Why and why not?
The covered languages support for \idx{execution models}{Execution model} is summarized in figure \ref{fig:first:phases:summary}. C and \csharp\ only support a single model each, while Elixir supports 5 (or 3 with variations). Does that mean that Elixir is a superior language? No, certainly not. But it does mean that you have more options as to how you execute Elixir code. Sometimes support for specific execution models matter, but often it does not.

\begin{figure}[tbp]
  \begin{center}
  \begin{tabular}{ccccl}
    \rotatebox{90}{\textbf{C}} & \rotatebox{90}{\textbf{\csharp}} & \rotatebox{90}{\textbf{Python}} & \rotatebox{90}{\textbf{Elixir}} & \textbf{Execution model} \\
    $\times$ & & & & Compile and execute directly on hardware \\
    & $\times$ & & $\times$ & Compile and execute in VM \\
    & & $\times$ & $\times$ & Script \\
    & & $\times$ & $\times$ & REPL \\
    & & $\times$ & $\times$ & Notebook \\
    & & & $\times$ & Script attached to running VM \\
    & & & $\times$ & REPL attached to running VM \\
    & & & $\times$ & Notebook attached to running VM \\
  \end{tabular}
\end{center}

  \caption{Summary of execution models of various programming languages.}
  \label{fig:first:phases:summary}
\end{figure}

% tradeoffs
It is also important to point out that the design of a programming language is riddled with \idx{tradeoffs}{Tradeoff}, and these are often resolved in very different ways. That results in some set of qualities, of which the supported set of execution models are found. As a developer, it will be your job to decide which combination of qualities are needed to maximize \idx{value}{Value} for a given project, and based on that choose a language.

\exercises{first}{The First Program}

