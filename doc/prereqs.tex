\chapter{Prereqs}

Before we get started on the subject matter, we need to install a number of programs that the remainder of the book will be referring to. You will need to use these (or similar) in the exercises. This may not seem like a particularly glorious task, but it is important to know the system in front of you and tasks like these are common for a software developer.

\section{Terminal and Shell}

\subsection{Installation}
\subsubsection{Windows}
\subsubsection{MacOS X}
\subsubsection{Debian Linux}
\subsection{Verifying Success}

\section{Text Editor}

% what is a text editor

% problem
While any text editor would do, not all are created equal. In particular, not all are equally suited for beginners. Some features help you think about the code, and understand it. Such features include \idx{syntax highlighting}{Syntax!Highlighting} (the highlighting of different parts of the code according to their role), \idx{line highlighting}{Line!Highlighting} and \idx{paranthesis highlighting}{Paranthesis!Highlighting}. Other features aim to maximize the rate of code production. Such features typically allows you to be very \say{productive} without having to think a whole lot about the code. This has a tendency to lead to \idx{absent-mindedness}{Absent-mindedness} and that is especially problematic when learning how to program. When working with the subject, you need to be present in mind and think about code. Always think about what makes it tick! Features that detracts from this are undesirable. These include anything that provides you with suggestions on how to write or rewrite your code. Any kind of \idx{autocomplete}{Autocomplete} functionality -- be it rooted in \idx{generative AI}{AI!Generative} or otherwise -- diverts your thought process from solving the problem proactively. That is bad. Other editors are very keen to suggest minor rewrites according to some set of more or less well-intended guidelines. However, these guidelines were rarely intended with novice developers in mind. In fact, then often hide complexity from the developer that is critical for the development of the novice developer.

% define subset of text editors that should be used, choose zed as first choice
So, we are looking for a text editor that has the positive aids and none of the negative aids (or at least provide easily available ways of disabling them). Seasoned developers may suggest \texttt{vi} or \texttt{emacs}, but they use their own shortcuts that have gone out of fashion. If you already use them, then there are no problems sticking to them. But everyone else should move on. Developers with deep roots in the industry will cling to \idxx{Visual Studio} (which is not portable) and have strong opinions on Visual Studio Code. Those opinions can go either way. Some companies are beginning to discard applicants who prefer \idxx{Visual Studio Code} as they have observed such developer to have a tendency to be way too reliant on the aids it provides. \idxx{Rider} is a popular option but recently they have gone all in on the problematic aids. This leaves a number of options. If you are on Linux, then you could use \texttt{gedit} or \texttt{kate}. If you are on Windows, then perhaps \texttt{notepad++} is a good option. However, the \texttt{zed}\footnote{\url{https://zed.dev}} editor is a good option, and it is available for all three major platforms. So, the following instructions will target \texttt{zed}. \texttt{zed} can be a bit finicky though, especially with regards to GPUs. So, if you are experiencing trouble, install one of the others.

\subsection{Installation}
\subsubsection{Windows}
\subsubsection{MacOS X}
\subsubsection{Debian Linux}
\subsection{Verifying Success}

\section{GIT}

\subsection{Installation}
\subsubsection{Windows}
\subsubsection{MacOS X}
\subsubsection{Debian Linux}
\subsection{Verifying Success}

\section{\csharp\ Development Environment}

\subsection{Installation}
\subsubsection{Windows}
\subsubsection{MacOS X}
\subsubsection{Debian Linux}
\subsection{Verifying Success}

\section{Livebook}

\subsection{Installation}
\subsubsection{Windows}
\subsubsection{MacOS X}
\subsubsection{Debian Linux}
\subsection{Verifying Success}
