\chapter{Software Development}

\section{Development Phases}
\subsection{Analysis}
\subsection{Design}
\subsection{Implementation}
\subsection{Evaluation}

\section{Development Models}
\subsection{Waterfall Model}

\begin{figure}[tbp]
  \begin{center}
  \begin{tikzpicture}[remember picture]
    \newcommand{\nodeheight}{7mm}
    \newcommand{\xstepsize}{6mm}
    \newcommand{\ystepsize}{-2mm}
    \tikzstyle{edge}  = [thick,>=stealth,draw=black]
    \tikzstyle{dedge} = [thick,->,>=stealth,draw=black]
    \tikzstyle{node}=[
      rectangle,
      draw=purple,
      anchor=north west,
      thick,
      minimum height=\nodeheight
    ]
    
    \node[node] (requirements)
      at (0,0)
      {Requirements};
    \node[node] (analysis)
      at ([xshift=\xstepsize, yshift=\ystepsize] requirements.south east)
      {Analysis};
    \node[node] (design)
      at ([xshift=\xstepsize, yshift=\ystepsize] analysis.south east)
      {Design};
    \node[node] (implementation)
      at ([xshift=\xstepsize, yshift=\ystepsize] design.south east)
      {Implementation};
    \node[node] (evaluation)
      at ([xshift=\xstepsize, yshift=\ystepsize] implementation.south east)
      {Evaluation};
    
    \draw[dedge] (requirements)-|([xshift=\nodeheight/2]analysis.north west);
    \draw[dedge] (analysis)-|([xshift=\nodeheight/2]design.north west);
    \draw[dedge] (design)-|([xshift=\nodeheight/2]implementation.north west);
    \draw[dedge] (implementation)-|([xshift=\nodeheight/2]evaluation.north west);
  \end{tikzpicture}
\end{center}

  \caption{Classical illustration of the waterfall model.}
  \label{fig:swdev:waterfall}
\end{figure}

\subsubsection{Critique}

\subsection{Iterative Models}


\begin{figure}[tbp]
  \begin{center}
  \begin{tikzpicture}[remember picture]
    \newcommand{\nodeheight}{7mm}
    \newcommand{\xstepsize}{6mm}
    \tikzstyle{edge}  = [thick,>=stealth,draw=black]
    \tikzstyle{dedge} = [thick,->,>=stealth,draw=black]
    \tikzstyle{node}=[
      rectangle,
      draw=purple,
      anchor=west,
      thick,
      minimum height=\nodeheight
    ]
    
    \node[node] (requirements)
      at (0,0)
      {Requirements};
    \node[node] (analysis)
      at ([xshift=\xstepsize] requirements.east)
      {Analysis};
    \node[node] (design)
      at ([xshift=\xstepsize] analysis.east)
      {Design};
    \node[node] (implementation)
      at ([xshift=\xstepsize] design.east)
      {Implementation};
    \node[node] (evaluation)
      at ([xshift=\xstepsize] implementation.east)
      {Evaluation};
    
    \draw[dedge] (requirements)--(analysis);
    \draw[dedge] (analysis)--(design);
    \draw[dedge] (design)--(implementation);
    \draw[dedge] (implementation)--(evaluation);
    
    \draw[dedge] (evaluation)--([yshift=-\nodeheight/2]evaluation.south)-|(requirements);
  \end{tikzpicture}
\end{center}

  \caption{Illustration of a typical iterative model.}
  \label{fig:swdev:waterfall}
\end{figure}

\subsubsection{Critique}

\section{Tools}
\subsection{Program Specification}
\subsection{Noun Verb Analysis}
\subsubsection{Critique}

% loosing the link between nouns and verbs

\subsection{CRC Cards}

\section{Reqirements}

\subsection{Functional Requirements}

\subsection{Nonfunctional Requirements}

\begin{inspiration}{\idx{Robert Love}{Love, Robert}\cite{love_guadec2005}}
  \quoted{Going to disk is 25 million times slower than hitting a general purpose register. Design accordingly.}
\end{inspiration}

\subsection{Difference}

% it can be hard to categorize

\subsection{Quality}

% what makes a good requirement? [testability]

\subsection{Organization}

% grouping/categorizing by quality attributes

\subsection{Critique}

% functional vs nonfunctional does not constitute a dichotomy: which qualities make the categorization easy? which qualities makes it hard? example of a requirement that is borderline, discusssion

% non-requirements or attractive qualities

\section{Local Maximum Trap}

% long-term planning: iterative approach for local maxima, long-term planning in order to avoid falling into the local maximum trap, getting something up and running as quick as possible is not the greatest success, it comes at a cost

\exercises{swdev}{Software Development}
