\chapter{Basic Datastructures}

% the problem of working with individual values

\section{Types}

\subsection{Complex Types}

\subsection{Reference Types}

\section{Sequences in Data}

\subsection{Arrays}

\subsubsection{Indexing}

% calculation, this is why most languages agree that arrays start at zero

\subsubsection{Multidimensiontal Arrays}

\subsection{Linked Lists}

% head and tail

\subsection{Doubly Linked Lists}

\subsection{Looping Sequences}

\subsection{Nesting}

\section{Structured Data}

\subsection{Colors}

In a computer, colors are typically represented using three values; one for red intensity, one for green intensity, and one for blue intensity. We say that these are the \textsl{components} of the color. There is a number of ways to express such an intensity. It seem natural to use a number between 0 and 100, or perhaps a number between 0.0 and 1.0. However, due to the way processors work on integers, these boundaries are not particularly special. Typically, a byte is use to represent each component of the color. That means that we have the ability to express $2^8=256$ different intensities of each component, or a total of $256^3=16777216$ different colors. The value zero is used to represent complete darkness. Full intensity is then the value 255.

% figure: color cube

% color models

\subsection{Points}

\section{Enumerations}

\subsection{State Machines}

\subsection{String Parsing Example}

