\chapter{Flow Control}

% motivation
So far, we have worked with a strict sequence of \idx{statements}{Statement}. All of these statements are evaluated, in \idx{order}{Order}. This is a quality that we will often come to rely on. But it is not expressive enough. We need to introduce \idx{choice}{Choice} and \idx{repetition}{Repetition} to our vocabulary, and our toolbox. Figure \ref{fig:flow:types} illustrates the set of basic flow abstractions that we will cover in this chapter. It is up to the individual programmer to decide how to combine these in order to implement desired behavior.

\begin{figure}[tbp]
  \begin{center}
  \begin{tikzpicture}[]
    \newcommand{\spacing}[0]{6mm}
    \newcommand{\absspacing}[0]{30mm}
    \newcommand{\seqX}[0]{ 0*\absspacing}
    \newcommand{\condX}[0]{1*\absspacing}
    \newcommand{\forkX}[0]{2*\absspacing}
    \newcommand{\whileX}[0]{3*\absspacing}
    \newcommand{\dowhileX}[0]{4*\absspacing}
    
    \tikzstyle{endpoint} = []
    \tikzstyle{stmt} = [circle,ultra thick,inner sep=0pt,minimum size=1.2mm,fill=none,draw=black]
    \tikzstyle{arrow} = [thick,->,>=stealth, draw=black]
    \tikzstyle{flow} = [arrow,ultra thick]
    \tikzstyle{label} = [anchor=north]
    \tikzstyle{subtype} = [draw=black]
    \tikzstyle{choice} = [draw=teal,fill=teal,minimum size=1.6mm]
    
    \coordinate (seqAnchor)  at (\seqX ,0);
    \coordinate (condAnchor) at (\condX,0);
    \coordinate (forkAnchor) at (\forkX,0);
    \coordinate (whileAnchor) at (\whileX,0);
    \coordinate (dowhileAnchor) at (\dowhileX,0);
    
    % sequence
    {
      \node[endpoint] (seqBegin) at ([yshift=0*\spacing]seqAnchor) {};
      \node[stmt] (seqA) at ([yshift=-1*\spacing]seqAnchor) {};
      \node[stmt] (seqB) at ([yshift=-2*\spacing]seqAnchor) {};
      \node[stmt] (seqC) at ([yshift=-3*\spacing]seqAnchor) {};
      \node[stmt] (seqD) at ([yshift=-4*\spacing]seqAnchor) {};
      \node[stmt] (seqE) at ([yshift=-5*\spacing]seqAnchor) {};
      \node[endpoint] (seqEnd) at ([yshift=-6.5*\spacing]seqAnchor) {};
      \node[label] () at ([yshift=0*\spacing]seqEnd) {Sequence};
      
      \path[flow] (seqBegin)
               -- (seqA)
               -- (seqB)
               -- (seqC)
               -- (seqD)
               -- (seqE)
               -- (seqEnd)
      ;
    }
    
    % conditional
    {
      \node[endpoint] (condBegin) at ([yshift=0*\spacing]condAnchor) {};
      \node[stmt] (condA) at ([yshift=-1*\spacing]condAnchor) {};
      \node[stmt,choice] (condB) at ([yshift=-2*\spacing]condAnchor) {};
      \node[stmt,draw=purple] (condC) at ([yshift=-3*\spacing,xshift=\spacing]condAnchor) {};
      \node[stmt,draw=purple] (condD) at ([yshift=-4*\spacing,xshift=\spacing]condAnchor) {};
      \node[stmt] (condE) at ([yshift=-5*\spacing]condAnchor) {};
      \node[endpoint] (condEnd) at ([yshift=-6.5*\spacing]condAnchor) {};
      \node[label] (cond) at ([yshift=0*\spacing]condEnd) {Conditional};
      
      \path[flow,-] (condBegin)
               -- (condA)
               -- (condB)
      ;
      \path[flow,-,draw=purple] (condB)
               -- (condE)
      ;
      \path[flow,-,draw=purple] (condB)
               to [out=270,in=90] (condC)
               -- (condD)
               to [out=270,in=90]  (condE)
      ;
      \path[flow] (condE)
               -- (condEnd)
      ;
    }
    
    % fork
    {
      \node[endpoint] (forkBegin) at ([yshift=0*\spacing]forkAnchor) {};
      \node[stmt] (forkA) at ([yshift=-1*\spacing]forkAnchor) {};
      \node[stmt,choice] (forkB) at ([yshift=-2*\spacing]forkAnchor) {};
      \node[stmt,draw=purple] (forkCtrue) at ([yshift=-3*\spacing,xshift=-\spacing]forkAnchor) {};
      \node[stmt,draw=purple] (forkDtrue) at ([yshift=-4*\spacing,xshift=-\spacing]forkAnchor) {};
      \node[stmt,draw=purple] (forkCfalse) at ([yshift=-3*\spacing,xshift=\spacing]forkAnchor) {};
      \node[stmt,draw=purple] (forkDfalse) at ([yshift=-4*\spacing,xshift=\spacing]forkAnchor) {};
      \node[stmt] (forkE) at ([yshift=-5*\spacing]forkAnchor) {};
      \node[endpoint] (forkEnd) at ([yshift=-6.5*\spacing]forkAnchor) {};
      \node[label] (fork) at ([yshift=0*\spacing]forkEnd) {Fork};
      
      \path[flow,-] (forkBegin)
               -- (forkA)
               -- (forkB)
      ;
      \path[flow,-,draw=purple] (forkB)
               to [out=270,in=90] (forkCtrue)
               -- (forkDtrue)
               to [out=270,in=90] (forkE)
      ;
      \path[flow,-,draw=purple] (forkB)
               to [out=270,in=90] (forkCfalse)
               -- (forkDfalse)
               to [out=270,in=90] (forkE)
      ;
      \path[flow] (forkE)
               -- (forkEnd)
      ;
    }
    
%    % while loop
%    {
%      \node[endpoint] (whileBegin) at ([yshift=0*\spacing]whileAnchor) {};
%      \node[stmt] (whileA) at ([yshift=-1*\spacing]whileAnchor) {};
%      \node[stmt,choice] (whileB) at ([yshift=-2*\spacing]whileAnchor) {};
%      \node[stmt,draw=purple] (whileC) at ([yshift=-3*\spacing]whileAnchor) {};
%      \node[stmt,draw=purple] (whileD) at ([yshift=-4*\spacing]whileAnchor) {};
%      \node[stmt] (whileE) at ([yshift=-5*\spacing,xshift=-\spacing]whileAnchor) {};
%      \node[stmt] (whileF) at ([yshift=-6*\spacing,xshift=-\spacing]whileAnchor) {};
%      \node[endpoint] (whileEnd) at ([yshift=-7.5*\spacing,xshift=-\spacing]whileAnchor) {};
%      \node[label] (while) at ([yshift=0*\spacing]whileEnd) {Skippable Loop};
%      
%      \path[flow] (whileBegin)
%               -- (whileA)
%               -- (whileB)
%               to [out=270,in=90] ([yshift=-1*\spacing,xshift=-1*\spacing]whileB)
%               -- (whileE)
%               -- (whileF)
%               -- (whileEnd)
%      ;
%      \path[flow,-,draw=purple] (whileB)
%               -- (whileC)
%               -- (whileD)
%               to [out=270,in=270] ([xshift=1*\spacing]whileD)
%               -- ([xshift=1*\spacing]whileB.center)
%               to [out=90,in=90] (whileB)
%      ;
%    }
    
    % while loop
    {
      \node[endpoint] (whileBegin) at ([yshift=0*\spacing]whileAnchor) {};
      \node[stmt] (whileA) at ([yshift=-1*\spacing]whileAnchor) {};
      \node[stmt] (whileB) at ([yshift=-2*\spacing]whileAnchor) {};
      \node[stmt,choice] (whileC) at ([yshift=-3*\spacing]whileAnchor) {};
      \node[stmt,draw=purple] (whileD) at ([yshift=-3.5*\spacing,xshift=0.5*\spacing]whileAnchor) {};
      \node[stmt,draw=purple] (whileE) at ([yshift=-3.5*\spacing,xshift=1.5*\spacing]whileAnchor) {};
      \node[stmt] (whileF) at ([yshift=-4*\spacing]whileAnchor) {};
      \node[stmt] (whileG) at ([yshift=-5*\spacing]whileAnchor) {};
      \node[endpoint] (whileEnd) at ([yshift=-6.5*\spacing]whileAnchor) {};
      \node[label] (while) at ([yshift=0*\spacing]whileEnd) {Skippable Loop};
      
      \path[flow] (whileBegin)
               -- (whileA)
               -- (whileB)
               -- (whileC)
               -- (whileF)
               -- (whileG)
               -- (whileEnd)
      ;
      \path[flow,-,draw=purple](whileD)
               -- (whileE)
               to [out=0,in=0,looseness=1.2] ([yshift=1*\spacing]whileE.east)
               -- ([yshift=1*\spacing]whileD.west)
               to [out=180,in=180,looseness=1.2] (whileD.west)
      ;
      
      % TODO: this is a hack to make this node appear on top of the loop path
      \node[stmt,choice] (whileC) at ([yshift=-3*\spacing]whileAnchor) {};
    }
    
    % do-while loop
    {
      \node[endpoint] (dowhileBegin) at ([yshift=0*\spacing]dowhileAnchor) {};
      \node[stmt] (dowhileA) at ([yshift=-1*\spacing]dowhileAnchor) {};
      \node[stmt,draw=purple] (dowhileB) at ([yshift=-2*\spacing]dowhileAnchor) {};
      \node[stmt,draw=purple] (dowhileC) at ([yshift=-3*\spacing]dowhileAnchor) {};
      \node[stmt,choice] (dowhileD) at ([yshift=-4*\spacing]dowhileAnchor) {};
      \node[stmt] (dowhileE) at ([yshift=-5*\spacing]dowhileAnchor) {};
      \node[endpoint] (dowhileEnd) at ([yshift=-6.5*\spacing]dowhileAnchor) {};
      \node[label] (dowhile) at ([yshift=0*\spacing]dowhileEnd) {Non-Skippable Loop};
      
      \path[flow] (dowhileBegin)
               -- (dowhileA)
               -- (dowhileB)
               -- (dowhileC)
               -- (dowhileD)
               -- (dowhileE)
               -- (dowhileEnd)
      ;
      \path[flow,-,draw=purple] (dowhileB)
               -- (dowhileC)
               -- (dowhileD)
               to [out=270,in=270,looseness=1.2] ([xshift=1*\spacing]dowhileD.south)
               -- ([xshift=1*\spacing]dowhileB.north)
               to [out=90,in=90,looseness=1.2] (dowhileB)
      ;
    }
    
    % branch
    \node[label] (branch) at ([yshift=-1*\spacing] $(cond)!0.5!(fork)$) {Branch};
    \path[subtype] (cond) |- (branch) -| (fork);
    
    % loop
    \node[label] (loop) at ([yshift=-1*\spacing] $(while)!0.5!(dowhile)$) {Repetition};
    \path[subtype] (while) |- (loop) -| (dowhile);
  \end{tikzpicture}
\end{center}

  \caption[Basic statement flow abstractions]{Basic statement flow abstractions. The execution path can be controlled by making a choice at runtime. The point of the choice is marked in \textcolor{teal}{teal} and the optional dynamic "routes" are marked in \textcolor{purple}{purple}.}
  \label{fig:flow:types}
\end{figure}

% context evolution: sequence (derived from the last statement), whichever way we go through the "track" we will only have access to the declarations from the common "ancestors"

\section{Choices in Logic}
\label{sec:flow:branch}

Hello

\subsection{\keywordname{if} and \keywordname{else}}
\subsection{\keywordname{unless}}
\subsection{Chaining Branches}
\subsection{\keywordname{switch} and \keywordname{case}}

\csharpsubsection{\csharp}

\begin{syntaxfloat}
  \section{Choices in Logic}
\label{sec:flow:branch}

Hello

\subsection{\keywordname{if} and \keywordname{else}}
\subsection{\keywordname{unless}}
\subsection{Chaining Branches}
\subsection{\keywordname{switch} and \keywordname{case}}

\csharpsubsection{\csharp}

\begin{syntaxfloat}
  \section{Choices in Logic}
\label{sec:flow:branch}

Hello

\subsection{\keywordname{if} and \keywordname{else}}
\subsection{\keywordname{unless}}
\subsection{Chaining Branches}
\subsection{\keywordname{switch} and \keywordname{case}}

\csharpsubsection{\csharp}

\begin{syntaxfloat}
  \input{syntax/flow_branch.tex}
  \caption{Statements for branching}
  \label{syntax:flow:branch}
\end{syntaxfloat}

\elixirsubsection{Elixir}
hello

\subsection{Blocks}

\subsubsection{Dangling Else}
\label{sec:flow:branch:danglingelse}

\csharpsubsection{\csharp}

\begin{syntaxfloat}
  \input{syntax/flow_block.tex}
  \caption{Block statements}
  \label{syntax:flow:block}
\end{syntaxfloat}


  \caption{Statements for branching}
  \label{syntax:flow:branch}
\end{syntaxfloat}

\elixirsubsection{Elixir}
hello

\subsection{Blocks}

\subsubsection{Dangling Else}
\label{sec:flow:branch:danglingelse}

\csharpsubsection{\csharp}

\begin{syntaxfloat}
  \begin{syntax}{stmt}
  \SyntaxWestSplit{MainWest}
  \SyntaxEastSplit{MainEast}
  
  \node[sequence] () at ([yshift=0*\syntaxruledist]$(begin)!0.5!(end)$) {
    \node[terminal]    (ruleIa) {\{};
    &
    \node[nonterminal] (ruleIb) {stmts};
    &
    \node[terminal]    (ruleIc) {\}};
    \\
  };
  
  \draw[path] (begin)--(ruleIa)--(ruleIb)--(ruleIc)--(end);
\end{syntax}

  \caption{Block statements}
  \label{syntax:flow:block}
\end{syntaxfloat}


  \caption{Statements for branching}
  \label{syntax:flow:branch}
\end{syntaxfloat}

\elixirsubsection{Elixir}
hello

\subsection{Blocks}

\subsubsection{Dangling Else}
\label{sec:flow:branch:danglingelse}

\csharpsubsection{\csharp}

\begin{syntaxfloat}
  \begin{syntax}{stmt}
  \SyntaxWestSplit{MainWest}
  \SyntaxEastSplit{MainEast}
  
  \node[sequence] () at ([yshift=0*\syntaxruledist]$(begin)!0.5!(end)$) {
    \node[terminal]    (ruleIa) {\{};
    &
    \node[nonterminal] (ruleIb) {stmts};
    &
    \node[terminal]    (ruleIc) {\}};
    \\
  };
  
  \draw[path] (begin)--(ruleIa)--(ruleIb)--(ruleIc)--(end);
\end{syntax}

  \caption{Block statements}
  \label{syntax:flow:block}
\end{syntaxfloat}


\chapter{Repetition}
\label{sec:flow:repetition}

Hello


\exercises{flow}{Flow Control}
