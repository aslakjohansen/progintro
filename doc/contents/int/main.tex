\chapter{Integers}
\label{sec:int}

% alt er tal, ikke altid lige åbenlyst, klasser af tal, den simpleste klasse er heltal
Alle værdier som vi ønsker at arbejde med i en computer er på en eller anden måde tal. Det er ikke altid lige åbentlyst at dette er tilfældet, men der er lang tid til at vi kommer til at støde på sådanne eksempler. Fra folkeskolen og gymnasiet kender vi til klasser af tal:
\begin{itemize}
  \descitem{De naturlige tal} ($\mathbb{N}$) Dette er de ikke-negative heltal\footnote{Der er to definitioner af de naturlige tal. Den anden definition er at det er de positive heltal; altså uden nul. Da denne definition er mindre anvendelig i computere bruger vi den fra brødteksten.}.
  \descitem{De hele tal} ($\mathbb{Z}$) Dette er alle positive og negative heltal samt nul.
  \descitem{De reelle tal} ($\mathbb{R}$) Dette er alle tal der kan placeres på en linje mellem $-\infty$ og $\infty$. Vi kalder dem ofte for kommatal.
  \descitem{De rationelle tal} ($\mathbb{Q}$) Dette er alle tal der kan skrives som en brøk $m/n$ hvor $m \in \mathbb{Z}$ og $m \in \mathbb{N}$.
  \descitem{De irrationelle tal} ($\mathbb{P}$) Dette er de tal som hører til $\mathbb{R}$ men ikke $\mathbb{Q}$. Det er her at $\pi$ og $e$ hører til.
\end{itemize}
Computere arbejder (typisk) med andre klasser af tal. I de følgende afsnit kommer vi til at se nærmere på de mest anvendte. Hvad der er fælles for dem er at de repræsenteres binært, altså ved hjælp af totalsystemet. Vi er vant til at arbejde med titalsystemet hvor et tal består af cifre mellem nul og ni. I totalsystemet består tal af cifre mellem nul og et (dvs. \texttt{0} eller \texttt{1}). Et sådant ciffer kaldes en \term{bit}, og en sekvens af 8 bits kaldes for en \term{byte}. Entalsformene er bit og byte, og flertalsformene er bits og bytes. Bits er for øvrigt SI enheden for information, må samme måde som at meter er SI enheden for distance. Vi måler med andre ord informationsmængde i bits.

\section{Integers}



\section{Floating Point Numbers}


\section{Boolean Values}

