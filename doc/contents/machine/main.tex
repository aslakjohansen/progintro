\chapter{Maskinen}
\label{sec:machine}

DECISION: MIPS or RISCV32I

\section{History}

% inspiration: http://andreasmhallberg.github.io/timeline-of-arab-grammarians/
{
\usetikzlibrary{calc, backgrounds}
\pgfdeclarelayer{background}
\pgfdeclarelayer{foreground}
\pgfsetlayers{background,main,foreground}

% vertical position of beginning of name
\newlength\nameh
  \setlength\nameh{6.1cm}                                                                                        
\def\startce{-200}
\def\fince{2020}

\newcommand\event[4][0]{
  % hame and yod on top
  \begin{pgfonlayer}{foreground}
    \path [draw] (#3,\nameh) ++ (#1,0) node [rotate=-90, anchor=west] (name) [fill=white, inner sep=0pt] {\makebox[2.5em][l]{#3}#2~\strut};
  \end{pgfonlayer}
    \draw [thin] (name.east) |- (#3,2) -- (#3,0) ;
    % Work below
    \node [font=\itshape, rotate=-90, anchor=west, yshift=#1] (work) at (#3,-2\baselineskip) {#4};
}

\begin{tikzpicture}[x=\textwidth/(\fince-\startce)]
        % events
        \event{Antikythera mechanism}{-87}{} % https://en.wikipedia.org/wiki/Antikythera_mechanism
        % tally stick
        \event{Jacquard machine}{1804}{} % https://en.wikipedia.org/wiki/Jacquard_machine
        % pascal/liebness
        \event{Difference Engine}{1822}{} % https://en.wikipedia.org/wiki/Difference_engine
        \event{Mutual Exclusion}{1965}{} % https://en.wikipedia.org/wiki/Mutual_exclusion
        \event{Programmeringssproget C}{1973}{} % https://en.wikipedia.org/wiki/C_(programming_language)
        
        \event{Programmeringssproget Java}{1995}{} % https://en.wikipedia.org/wiki/Java_(programming_language)
        
        % Alpacaen tæmmes
        
        % axis
        \draw [->, very thick] (\startce,0) -- (\fince,0);

        % axis ticks
        \foreach \x in {-200,0,...,\fince}
            {\draw (\x, 0) -- (\x,-.2) node [anchor=north, font=\bfseries,inner sep=0pt] {\x\strut};}

\end{tikzpicture}
}

\section{Registre}

\section{Hukommelsesmodel}

\section{Instruktioner}

\subsection{Aritmetriske Operationer}

\subsection{Valg}


