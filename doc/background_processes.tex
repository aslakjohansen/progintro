\section{Processes}

Seen from the perspective of the operating system, the stuff that runs on your computer is split into \idx{processes}{Process}. Sometimes an application will consist of multiple processes, but most of the time there will be one process per application. But there are other processes. Some perform services under the hood, and others handle the interaction between the user and the operating system.

\subsection{Kernel}

Any general-purpose operating system will have a \idx{kernel}{Kernel} at its core. The kernel is a gatekeeper of pretty much every central operation. It is involved whenever the computer receives input and whenever a process want to create an output. It is where processes are created, stopped and given time.

% time
Processes are typically written as if nothing else is running on the machine. But in reality, lots of processes are running at any given time. In fact, while I am writing this text, I have 643 processes running on my laptop. Some of them are \idx{active}{Process!Active}, and others are \idx{waiting}{Process!Waiting} for some kind of input. The kernel is making sure that each of the active ones get time (on the processor) on a regular basis. As other processes need \idx{time}{Time} as well, this also means that there are interruptions in the time given to each process. This in term means that the processes essentially experience gaps in time where they are frozen.

\subsection{Desktop Environment}

% what: one or more processes, drawing on screen, arranging windows, starting applications, services for common configuration and communication
Most people interact with their computer through a desktop environment. \defi{Desktop environments}{Desktop environment} are comprised of a number of processes responsible for drawing on the \idx{screen}{Screen}, arranging windows, starting applications, services for shared configuration and communication between applications.

% examples: windows, osx, linux (gnome, kde, ...)
Examples of desktop environments include Microsoft \idx{Windows}{Windows}, Apple \idx{OSX}{OSX}, \idx{Gnome}{Gnome} and \idx{KDE}{KDE}. The latter two are popular choices on \idx{Linux}{Linux}, but there are many other options.

\subsection{Shell}

\begin{figure}[tbp]
  \begin{center}
  \begin{tikzpicture}[remember picture]
    \tikzstyle{dir} = [
      rectangle,
      draw,
      anchor=north,
      align=center,
    ]
    \tikzstyle{arrow} = [
      thick,
      ->,
      >=stealth,
      draw=black,
    ]
    
    \newcommand{\padding}[0]{1cm}
    \newcommand{\concretecommand}[2]{node[midway,#1] {\scalebox{0.7}{\texttt{\textcolor{purple}{#2}}}}}
    
    \coordinate (origo) at (0,0);
    
    \node[dir,fill=black,text=white] (wd) at (origo) {current directory};
    \node[dir,anchor=south] (parent) at ([yshift=\padding] wd.north) {\textsl{parent}};
    \node[dir] (childII) at ([yshift=-\padding] wd.south) {\textsl{child2}};
    \node[dir,anchor=east] (childI) at ([xshift=-\padding] childII.west) {\textsl{child1}};
    \node[dir,anchor=west] (childIII) at ([xshift=\padding] childII.east) {\textsl{child3}};
    
    \draw[arrow] (wd)--(parent) node[midway,right] {\scalebox{0.7}{\texttt{\textcolor{purple}{cd ..}}}};
    \draw[arrow] (wd)--(childI)  node[midway,left] {\scalebox{0.7}{\texttt{\textcolor{purple}{cd child1}}}};
    \draw[arrow] (wd)--(childII)  node[midway,right] {\scalebox{0.7}{\texttt{\textcolor{purple}{?}}}};
    \draw[arrow] (wd)--(childIII)  node[midway,right] {\scalebox{0.7}{\texttt{\textcolor{purple}{cd child3}}}};
  \end{tikzpicture}
\end{center}

  \caption{Navigating the filesystem.}
  \label{fig:bg:processes:navigation}
\end{figure}

\subsection{Working with Files}

\subsubsection{Copying Files}

% what does it mean to copy a file?

% explain it

\begin{figure}[tbp]
  \begin{center}
  \begin{tikzpicture}[remember picture]
    \tikzstyle{title} = [
      anchor=south,
      align=center,
    ]
    \tikzstyle{arrow} = [
      thick,
      ->,
      >=stealth,
      draw=black,
    ]
    \tikzstyle{fileline}=[
      arrow,
      ultra thick,
      draw=black!20,
    ]
    \tikzstyle{fd} = [
      anchor=center,
      align=center,
      fill=black!20,
      minimum width=2cm,
      minimum height=0.6cm,
    ]
    \tikzstyle{data} = [
      anchor=center,
      fill=black,
      minimum width=1.5mm,
      minimum height=1.5mm,
    ]
    
    \newcommand{\shiftdistance}[0]{5cm}
    \newcommand{\figheight}[0]{5.2cm}
    
    \coordinate (memAnchor) at (0,-1cm);
    \coordinate (srcAnchor) at ([xshift=-\shiftdistance, yshift=0] memAnchor);
    \coordinate (dstAnchor) at ([xshift= \shiftdistance, yshift=0] memAnchor);
    
    % src line
    {
      \node[title] (srcTitle) at ([yshift=2mm] srcAnchor) {Source\\File};
      \draw[fileline] (srcAnchor) -- ([yshift=-\figheight] srcAnchor);
    }
    
    % dst line
    {
      \node[title] (dstTitle) at ([yshift=2mm] dstAnchor) {Destinations\\File};
      \draw[fileline] (dstAnchor) -- ([yshift=-\figheight] dstAnchor);
    }
    
    % mem line
    {
      \node[title] (memTitle) at ([yshift=2mm] memAnchor) {Program\\Memory};
      \draw[fileline] (memAnchor) -- ([yshift=-\figheight] memAnchor);
    }
    
    % open(src)
    {
      \node[fd] (srcOpen) at ([yshift=-4mm] srcAnchor) {\small{open file}};
    }
    
    % open(dst)
    {
      \node[fd] (dstOpen) at ([yshift=-11mm] dstAnchor) {\small{open file}};
    }
    
    % buffer
    {
      \node[fd, minimum width=0.6cm, minimum height=1.4cm] (buffer) at ([yshift=-25mm] memAnchor) {\rotatebox{90}{\small{buffer}}};
    }
    
    % read
    {
      \node[data] (srcDataI)   at ([yshift=-19mm] srcAnchor) {};
      \node[data] (srcDataII)  at ([yshift=-3.2mm] srcDataI) {};
      \node[data] (srcDataIII) at ([yshift=-3.2mm] srcDataII) {};
      
      \draw[arrow, dashed, draw=blue]
        ([xshift=4mm] srcDataII.center)--([xshift=\shiftdistance-0.3cm] srcDataII.center)
        node[midway,sloped,above]
        {\scalebox{0.7}{\textsl{\textcolor{blue}{read}}}};
    }
    
    % write
    {
      \node[data] (dstDataIII) at ([yshift=-31mm] dstAnchor) {};
      \node[data] (dstDataII)  at ([yshift=3.2mm] dstDataIII) {};
      \node[data] (dstDataI)   at ([yshift=3.2mm] dstDataII) {};
      
      \draw[arrow, dashed, draw=blue]
        ([xshift=-\shiftdistance+0.3cm] dstDataII.center)--([xshift=-4mm] dstDataII.center)
        node[midway,sloped,above]
        {\scalebox{0.7}{\textsl{\textcolor{blue}{write}}}};
    }
    
    % repetition
    {
      \node[rectangle, thick, dashed, draw=purple, anchor=center, minimum width=2*\shiftdistance+10mm, minimum height=1.8cm] (repetition) at ([yshift=-25mm] memAnchor) {};
      \node[] (repetitionLabel) at ([xshift=-2mm] repetition.west) {\rotatebox{90}{\scalebox{0.60}{\textsl{\textcolor{purple}{repeat as needed}}}}};
    }
    
    % close(src)
    {
      \node[fd] (srcOpen) at ([yshift=-39mm] srcAnchor) {\small{close file}};
    }
    
    % close(dst)
    {
      \node[fd] (dstOpen) at ([yshift=-46mm] dstAnchor) {\small{close file}};
    }
  \end{tikzpicture}
\end{center}

  \caption{The process of copying a file.}
  \label{fig:bg:processes:copy}
\end{figure}

\subsubsection{Editing Files}

% what does it mean to edit a file?

% explain it
Figure \ref{fig:bg:processes:edit} illustrates how a typical \idx{text editor}{Text editor} will work with a file. First the file is \idx{opened}{File!Opening}. Then, the contents of the file is read and written to \idx{primary memory}{Memory!Primary}. The user may then manipulate the in-memory representation through some user interface, and saved the modified buffer (memory representation) to disk. Finally, the program will \idx{close}{File!Closing} the file.

\begin{figure}[tbp]
  \begin{center}
  \begin{tikzpicture}[remember picture]
    \tikzstyle{title} = [
      anchor=south,
      align=center,
    ]
    \tikzstyle{arrow} = [
      thick,
      ->,
      >=stealth,
      draw=black,
    ]
    \tikzstyle{fileline}=[
      arrow,
      ultra thick,
      draw=black!20,
    ]
    \tikzstyle{fd} = [
      anchor=center,
      align=center,
      fill=black!20,
      minimum width=2cm,
      minimum height=0.6cm,
    ]
    \tikzstyle{data} = [
      anchor=center,
      fill=black,
      minimum width=1.5mm,
      minimum height=1.5mm,
    ]
    
    \newcommand{\shiftdistance}[0]{5cm}
    \newcommand{\moddistance}[0]{6mm}
    \newcommand{\figheight}[0]{5.2cm}
    
    \coordinate (memAnchor) at (0,-1cm);
    \coordinate (userAnchor) at ([xshift=-\shiftdistance, yshift=0] memAnchor);
    \coordinate (fileAnchor) at ([xshift= \shiftdistance, yshift=0] memAnchor);
    
    % user line
    {
      \node[title] (userTitle) at ([yshift=2mm] userAnchor) {User\\Interface};
      \draw[fileline] (userAnchor) -- ([yshift=-\figheight] userAnchor);
    }
    
    % file line
    {
      \node[title] (fileTitle) at ([yshift=2mm] fileAnchor) {File of\\Interest};
      \draw[fileline] (fileAnchor) -- ([yshift=-\figheight] fileAnchor);
    }
    
    % mem line
    {
      \node[title] (memTitle) at ([yshift=2mm] memAnchor) {Program\\Memory};
      \draw[fileline] (memAnchor) -- ([yshift=-\figheight] memAnchor);
    }
    
    % open(file)
    {
      \node[fd] (fileOpen) at ([yshift=-4mm] fileAnchor) {\small{open file}};
    }
    
    % buffer
    {
      \node[fd, minimum width=0.6cm, minimum height=2.8cm] (buffer) at ([yshift=-25mm] memAnchor) {\rotatebox{90}{\small{buffer}}};
    }
    
    % read
    {
      \node[data, anchor=north] (dataI)   at ([yshift=-2mm] fileOpen.south) {};
      \node[data] (dataII)  at ([yshift=-3.2mm] dataI) {};
      \node[data] (dataIII) at ([yshift=-3.2mm] dataII) {};
      
      \draw[arrow, dashed, draw=blue]
        ([xshift=-4mm] dataII.center)--([xshift=-\shiftdistance+0.3cm] dataII.center)
        node[midway,sloped,above]
        {\scalebox{0.7}{\textsl{\textcolor{blue}{read}}}};
    }
    
    % modify 1
    {
      \draw[arrow, dashed, draw=blue]
        ([xshift=-\shiftdistance, yshift=\moddistance] buffer.center)--([xshift=-0.3cm, yshift=\moddistance] buffer.center)
        node[midway,sloped,above]
        {\scalebox{0.7}{\textsl{\textcolor{blue}{modify}}}};
    }
    
    % modify 2
    {
      \draw[arrow, dashed, draw=blue]
        ([xshift=-\shiftdistance, yshift=0mm] buffer.center)--([xshift=-0.3cm, yshift=0mm] buffer.center)
        node[midway,sloped,above]
        {\scalebox{0.7}{\textsl{\textcolor{blue}{modify}}}};
    }
    
    % modify 3
    {
      \draw[arrow, dashed, draw=blue]
        ([xshift=-\shiftdistance, yshift=-\moddistance] buffer.center)--([xshift=-0.3cm, yshift=-\moddistance] buffer.center)
        node[midway,sloped,above]
        {\scalebox{0.7}{\textsl{\textcolor{blue}{modify}}}};
    }
    
    % write
    {
      \node[data] (dstDataIII) at ([yshift=-40mm] fileAnchor) {};
      \node[data] (dstDataII)  at ([yshift=3.2mm] dstDataIII) {};
      \node[data] (dstDataI)   at ([yshift=3.2mm] dstDataII) {};
      
      \draw[arrow, dashed, draw=blue]
        ([xshift=-\shiftdistance+0.3cm] dstDataII.center)--([xshift=-4mm] dstDataII.center)
        node[midway,sloped,above]
        {\scalebox{0.7}{\textsl{\textcolor{blue}{write}}}};
    }
    
    % close(file)
    {
      \node[fd] (fileClose) at ([yshift=-46mm] fileAnchor) {\small{close file}};
    }
  \end{tikzpicture}
\end{center}

  \caption{The process of editing a file.}
  \label{fig:bg:processes:edit}
\end{figure}

