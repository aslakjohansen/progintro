\section{Processes}

Seen from the perspective of the operating system, the stuff that runs on your computer is split into \idx{processes}{Process}. Sometimes an application will consist of multiple processes, but most of the time there will be one process per application. But there are other processes. Some perform services under the hood, and others handle the interaction between the user and the operating system.

\subsection{Kernel}

Any general-purpose operating system will have a \idx{kernel}{Kernel} at its core. The kernel is a gatekeeper of pretty much every central operation. It is involved whenever the computer receives input and whenever a process want to create an output. It is where processes are created, stopped and given time.

% time
Processes are typically written as if nothing else is running on the machine. But in reality, lots of processes are running at any given time. In fact, while I am writing this text, I have 643 processes running on my laptop. Some of them are \idx{active}{Process!Active}, and others are \idx{waiting}{Process!Waiting} for some kind of input. The kernel is making sure that each of the active ones get time (on the processor) on a regular basis. As other processes need \idx{time}{Time} as well, this also means that there are interruptions in the time given to each process. This in term means that the processes essentially experience gaps in time where they are frozen.

\subsection{Desktop Environment}

% what: one or more processes, drawing on screen, arranging windows, starting applications, services for common configuration and communication
Most people interact with their computer through a desktop environment. \defi{Desktop environments}{Desktop environment} are comprised of a number of processes responsible for drawing on the \idx{screen}{Screen}, arranging windows, starting applications, services for shared configuration and communication between applications.

% examples: windows, osx, linux (gnome, kde, ...)
Examples of desktop environments include Microsoft \idx{Windows}{Windows}, Apple \idx{OSX}{OSX}, \idx{Gnome}{Gnome} and \idx{KDE}{KDE}. The latter two are popular choices on \idx{Linux}{Linux}, but there are many other options. Desktop environments are sometimes referred to as \textsl{desktop shells}.

\subsection{File Manager}

\subsection{Terminal}

One type of application that you will get familiar with is that of a \defi{terminal}{Terminal}. In some ways, this is an extremely simple application. It opens a window that can act as the interface to a \idx{text-based}{Program!Text-based} program. It allows you to provide inputs to that program, and see the output that program prints to the \textsl{screen}. Sometimes a terminal is called a \say{console} or a \say{command prompt}. The latter term is a bit unfortunate though as it also encompasses a \textsl{shell}.

\subsection{Shell}

The typical program to run inside a terminal is a \defi{shell}{Shell}. This is an interactive program that provides you with a \idx{prompt}{Prompt}. Through the prompt you can issue commands. Some commands change the \idx{working directory}{Directory!Working}, some adjust \idx{environment variables}{Environment variable} (i.e., a set of named values that affect the way programs execute), and some start other programs. Figure \ref{fig:bg:processes:navigation} illustrates how one can use the \commandname{cd} command to change the working directory.

\begin{figure}[tbp]
  \begin{center}
  \begin{tikzpicture}[remember picture]
    \tikzstyle{dir} = [
      rectangle,
      draw,
      anchor=north,
      align=center,
    ]
    \tikzstyle{arrow} = [
      thick,
      ->,
      >=stealth,
      draw=black,
    ]
    
    \newcommand{\padding}[0]{1cm}
    \newcommand{\concretecommand}[2]{node[midway,#1] {\scalebox{0.7}{\texttt{\textcolor{purple}{#2}}}}}
    
    \coordinate (origo) at (0,0);
    
    \node[dir,fill=black,text=white] (wd) at (origo) {current directory};
    \node[dir,anchor=south] (parent) at ([yshift=\padding] wd.north) {\textsl{parent}};
    \node[dir] (childII) at ([yshift=-\padding] wd.south) {\textsl{child2}};
    \node[dir,anchor=east] (childI) at ([xshift=-\padding] childII.west) {\textsl{child1}};
    \node[dir,anchor=west] (childIII) at ([xshift=\padding] childII.east) {\textsl{child3}};
    
    \draw[arrow] (wd)--(parent) node[midway,right] {\scalebox{0.7}{\texttt{\textcolor{purple}{cd ..}}}};
    \draw[arrow] (wd)--(childI)  node[midway,left] {\scalebox{0.7}{\texttt{\textcolor{purple}{cd child1}}}};
    \draw[arrow] (wd)--(childII)  node[midway,right] {\scalebox{0.7}{\texttt{\textcolor{purple}{?}}}};
    \draw[arrow] (wd)--(childIII)  node[midway,right] {\scalebox{0.7}{\texttt{\textcolor{purple}{cd child3}}}};
  \end{tikzpicture}
\end{center}

  \caption{Navigating the filesystem.}
  \label{fig:bg:processes:navigation}
\end{figure}

Different shells will have different names for commands that perform similar action. In the book, we follow the convention of the \idx{bash}{Bash} shell (that is often default on Linux). The commands that we cover are called the same in \idx{zsh}{Zsh} (that is default on OSX). Commonly used commands include:
\begin{itemize}
  \item \highlight{\texttt{ls}} (\textsl{list}) This command lists the contents of a directory. A few variants:
    \begin{itemize}
      \item \highlight{\texttt{ls -l}} Also display select metadata information for each file.
      \item \highlight{\texttt{ls ..}} List contents for parent directory (any valid path can be given as parameter).
    \end{itemize}
  \item \highlight{\texttt{rm}} (\textsl{remove}) Remove a subtree (\idx{unlink}{Unlink} really) of the filesystem. Look it up before you delete anything of value.
  \item \highlight{\texttt{mkdir}} (\textsl{make directory}) Create a new directory:
    \begin{itemize}
      \item \highlight{\texttt{mkdir mydir}} Creates a directory in the current working directory.
    \end{itemize}
  \item \highlight{\texttt{cp}} (\textsl{copy}) copy a subtree of the filesystem (by default only a single element):
    \begin{itemize}
      \item \highlight{\texttt{cp source destination}} Creates a copy of the source under the name of the destination.
      \item \highlight{\texttt{cp -r source destination}} Same, but recursively.
    \end{itemize}
  \item \highlight{\texttt{mv}} (\textsl{move}) move a subtree of the filesystem :
    \begin{itemize}
      \item \highlight{\texttt{cp source destination}} Move the source to the destination.
    \end{itemize}
\end{itemize}

% built-in help function
Most commands will \idx{accept}{Accept!Argument} a \highlight{\texttt{--help}} and/or \highlight{\texttt{-h}} \idx{argument}{Argument} and react to it by printing out a help message instead of completing the operation it was designed to do. These arguments are also called \idx{flags}{Flag}.

\subsection{Working with Files}

Files are stored in a \idx{filesystem}{Filesystem} that is usually persisted on a disk. This works out very well when you \idx{reboot}{Reboot} your computer and see that your data is still there. That would not have been the case had the files been stored in \idx{main memory}{Memory!Main}. However, the computer can only work with data in main memory, so in order to do something with the contents of those files, they will have to be read into memory. And if you want the result of what you end up doing with them to survive a reboot, then you need to store it on the disk again.

\subsubsection{Copying Files}

% what does it mean to copy a file?
Creating a \idx{copy}{Copy} of a file means to replicate the contents and the metadata of the file in some other directory of a filesystem. That could happen on the same disk or another disk, and that disk could be on the same or another computer.

% explain it
The principal operation is illustrated in figure \ref{fig:bg:processes:copy}. First the source file is opened for reading and the destination file for writing. This gives the process a \idx{file descriptor}{File descriptor} for each through which it can ask the kernel to perform operations on the file. The process performing the copying \idx{allocates}{Allocation} a \idx{buffer}{Buffer} of a reasonable size. Then it reads data from the source file to the buffer. When this operation is completed it writes the contents of the buffer to the destination file. The buffer will likely be smaller than the file being copied (otherwise you would need to have more memory than the largest file), so the process may have to repeated a number of times with different \idx{offsets}{Offset}. When this process has been repeated enough times, the source and destination files are closed.

\begin{figure}[tbp]
  \begin{center}
  \begin{tikzpicture}[remember picture]
    \tikzstyle{title} = [
      anchor=south,
      align=center,
    ]
    \tikzstyle{arrow} = [
      thick,
      ->,
      >=stealth,
      draw=black,
    ]
    \tikzstyle{fileline}=[
      arrow,
      ultra thick,
      draw=black!20,
    ]
    \tikzstyle{fd} = [
      anchor=center,
      align=center,
      fill=black!20,
      minimum width=2cm,
      minimum height=0.6cm,
    ]
    \tikzstyle{data} = [
      anchor=center,
      fill=black,
      minimum width=1.5mm,
      minimum height=1.5mm,
    ]
    
    \newcommand{\shiftdistance}[0]{5cm}
    \newcommand{\figheight}[0]{5.2cm}
    
    \coordinate (memAnchor) at (0,-1cm);
    \coordinate (srcAnchor) at ([xshift=-\shiftdistance, yshift=0] memAnchor);
    \coordinate (dstAnchor) at ([xshift= \shiftdistance, yshift=0] memAnchor);
    
    % src line
    {
      \node[title] (srcTitle) at ([yshift=2mm] srcAnchor) {Source\\File};
      \draw[fileline] (srcAnchor) -- ([yshift=-\figheight] srcAnchor);
    }
    
    % dst line
    {
      \node[title] (dstTitle) at ([yshift=2mm] dstAnchor) {Destinations\\File};
      \draw[fileline] (dstAnchor) -- ([yshift=-\figheight] dstAnchor);
    }
    
    % mem line
    {
      \node[title] (memTitle) at ([yshift=2mm] memAnchor) {Program\\Memory};
      \draw[fileline] (memAnchor) -- ([yshift=-\figheight] memAnchor);
    }
    
    % open(src)
    {
      \node[fd] (srcOpen) at ([yshift=-4mm] srcAnchor) {\small{open file}};
    }
    
    % open(dst)
    {
      \node[fd] (dstOpen) at ([yshift=-11mm] dstAnchor) {\small{open file}};
    }
    
    % buffer
    {
      \node[fd, minimum width=0.6cm, minimum height=1.4cm] (buffer) at ([yshift=-25mm] memAnchor) {\rotatebox{90}{\small{buffer}}};
    }
    
    % read
    {
      \node[data] (srcDataI)   at ([yshift=-19mm] srcAnchor) {};
      \node[data] (srcDataII)  at ([yshift=-3.2mm] srcDataI) {};
      \node[data] (srcDataIII) at ([yshift=-3.2mm] srcDataII) {};
      
      \draw[arrow, dashed, draw=blue]
        ([xshift=4mm] srcDataII.center)--([xshift=\shiftdistance-0.3cm] srcDataII.center)
        node[midway,sloped,above]
        {\scalebox{0.7}{\textsl{\textcolor{blue}{read}}}};
    }
    
    % write
    {
      \node[data] (dstDataIII) at ([yshift=-31mm] dstAnchor) {};
      \node[data] (dstDataII)  at ([yshift=3.2mm] dstDataIII) {};
      \node[data] (dstDataI)   at ([yshift=3.2mm] dstDataII) {};
      
      \draw[arrow, dashed, draw=blue]
        ([xshift=-\shiftdistance+0.3cm] dstDataII.center)--([xshift=-4mm] dstDataII.center)
        node[midway,sloped,above]
        {\scalebox{0.7}{\textsl{\textcolor{blue}{write}}}};
    }
    
    % repetition
    {
      \node[rectangle, thick, dashed, draw=purple, anchor=center, minimum width=2*\shiftdistance+10mm, minimum height=1.8cm] (repetition) at ([yshift=-25mm] memAnchor) {};
      \node[] (repetitionLabel) at ([xshift=-2mm] repetition.west) {\rotatebox{90}{\scalebox{0.60}{\textsl{\textcolor{purple}{repeat as needed}}}}};
    }
    
    % close(src)
    {
      \node[fd] (srcOpen) at ([yshift=-39mm] srcAnchor) {\small{close file}};
    }
    
    % close(dst)
    {
      \node[fd] (dstOpen) at ([yshift=-46mm] dstAnchor) {\small{close file}};
    }
  \end{tikzpicture}
\end{center}

  \caption{The process of copying a file.}
  \label{fig:bg:processes:copy}
\end{figure}

\subsubsection{Editing Files}

% what does it mean to edit a file?
When we edit a file, we change the contents of that file. If we add something to the beginning, then the old contents needs to be shifted forward to make space for the new material. If we add something at the end, then some material can be easily added. If the modification happens anywhere in between then part of the original contents needs to be moved. Doing these operations on disk would be highly ineffective. So, in reality it happens in the computers memory and the saving to disk is initiated by either a human intervention or something like a \idx{timeout}{Timeout}.

% explain it
Figure \ref{fig:bg:processes:edit} illustrates how a typical \idx{text editor}{Text editor} will work with a file. First the file is \idx{opened}{File!Opening}. Then, the contents of the file is read and written to \idx{primary memory}{Memory!Primary}. The user may then manipulate the in-memory representation through some user interface, and saved the modified buffer (memory representation) to disk. Finally, the program will \idx{close}{File!Closing} the file.

\begin{figure}[tbp]
  \begin{center}
  \begin{tikzpicture}[remember picture]
    \tikzstyle{title} = [
      anchor=south,
      align=center,
    ]
    \tikzstyle{arrow} = [
      thick,
      ->,
      >=stealth,
      draw=black,
    ]
    \tikzstyle{fileline}=[
      arrow,
      ultra thick,
      draw=black!20,
    ]
    \tikzstyle{fd} = [
      anchor=center,
      align=center,
      fill=black!20,
      minimum width=2cm,
      minimum height=0.6cm,
    ]
    \tikzstyle{data} = [
      anchor=center,
      fill=black,
      minimum width=1.5mm,
      minimum height=1.5mm,
    ]
    
    \newcommand{\shiftdistance}[0]{5cm}
    \newcommand{\moddistance}[0]{6mm}
    \newcommand{\figheight}[0]{5.2cm}
    
    \coordinate (memAnchor) at (0,-1cm);
    \coordinate (userAnchor) at ([xshift=-\shiftdistance, yshift=0] memAnchor);
    \coordinate (fileAnchor) at ([xshift= \shiftdistance, yshift=0] memAnchor);
    
    % user line
    {
      \node[title] (userTitle) at ([yshift=2mm] userAnchor) {User\\Interface};
      \draw[fileline] (userAnchor) -- ([yshift=-\figheight] userAnchor);
    }
    
    % file line
    {
      \node[title] (fileTitle) at ([yshift=2mm] fileAnchor) {File of\\Interest};
      \draw[fileline] (fileAnchor) -- ([yshift=-\figheight] fileAnchor);
    }
    
    % mem line
    {
      \node[title] (memTitle) at ([yshift=2mm] memAnchor) {Program\\Memory};
      \draw[fileline] (memAnchor) -- ([yshift=-\figheight] memAnchor);
    }
    
    % open(file)
    {
      \node[fd] (fileOpen) at ([yshift=-4mm] fileAnchor) {\small{open file}};
    }
    
    % buffer
    {
      \node[fd, minimum width=0.6cm, minimum height=2.8cm] (buffer) at ([yshift=-25mm] memAnchor) {\rotatebox{90}{\small{buffer}}};
    }
    
    % read
    {
      \node[data, anchor=north] (dataI)   at ([yshift=-2mm] fileOpen.south) {};
      \node[data] (dataII)  at ([yshift=-3.2mm] dataI) {};
      \node[data] (dataIII) at ([yshift=-3.2mm] dataII) {};
      
      \draw[arrow, dashed, draw=blue]
        ([xshift=-4mm] dataII.center)--([xshift=-\shiftdistance+0.3cm] dataII.center)
        node[midway,sloped,above]
        {\scalebox{0.7}{\textsl{\textcolor{blue}{read}}}};
    }
    
    % modify 1
    {
      \draw[arrow, dashed, draw=blue]
        ([xshift=-\shiftdistance, yshift=\moddistance] buffer.center)--([xshift=-0.3cm, yshift=\moddistance] buffer.center)
        node[midway,sloped,above]
        {\scalebox{0.7}{\textsl{\textcolor{blue}{modify}}}};
    }
    
    % modify 2
    {
      \draw[arrow, dashed, draw=blue]
        ([xshift=-\shiftdistance, yshift=0mm] buffer.center)--([xshift=-0.3cm, yshift=0mm] buffer.center)
        node[midway,sloped,above]
        {\scalebox{0.7}{\textsl{\textcolor{blue}{modify}}}};
    }
    
    % modify 3
    {
      \draw[arrow, dashed, draw=blue]
        ([xshift=-\shiftdistance, yshift=-\moddistance] buffer.center)--([xshift=-0.3cm, yshift=-\moddistance] buffer.center)
        node[midway,sloped,above]
        {\scalebox{0.7}{\textsl{\textcolor{blue}{modify}}}};
    }
    
    % write
    {
      \node[data] (dstDataIII) at ([yshift=-40mm] fileAnchor) {};
      \node[data] (dstDataII)  at ([yshift=3.2mm] dstDataIII) {};
      \node[data] (dstDataI)   at ([yshift=3.2mm] dstDataII) {};
      
      \draw[arrow, dashed, draw=blue]
        ([xshift=-\shiftdistance+0.3cm] dstDataII.center)--([xshift=-4mm] dstDataII.center)
        node[midway,sloped,above]
        {\scalebox{0.7}{\textsl{\textcolor{blue}{write}}}};
    }
    
    % close(file)
    {
      \node[fd] (fileClose) at ([yshift=-46mm] fileAnchor) {\small{close file}};
    }
  \end{tikzpicture}
\end{center}

  \caption{The process of editing a file.}
  \label{fig:bg:processes:edit}
\end{figure}

