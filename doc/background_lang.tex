\section{Programming Languages}
\label{sec:lang}

\begin{inspiration}{\idx{Larry Wall}{Wall, Larry}\cite{progLangJurassicPark}}
  \quoted{It might seem easy enough, but computer language design is just like a stroll in the park. Jurassic Park, that is.}
\end{inspiration}

\begin{figure}[tbp]
  \centering

% inspiration: https://tex.stackexchange.com/questions/309425/how-is-it-possible-to-make-a-magic-quadrant
\begin{tikzpicture}[squares/.style={align=center, text width=3cm, minimum width=4cm, minimum height=4cm}]
  \node[squares,fill=lightblue] (A) at (0,0) {Python\\R\\JavaScript};
  \node[squares,fill=darkblue,anchor=west] (B) at (A.east) {Java\\\csharp\\Elixir\\Go};
  \node[squares,fill=darkblue,anchor=north] (C) at (A.south){};
  \node[squares,fill=lightblue,anchor=north] (D) at (B.south) {Assembly\\C\\Rust};
  \node[inner sep=0pt,draw=grey,ultra thick,rounded corners=2pt,fit=(A)(B)(C)(D)] {}; 
  
  \node[anchor=east,xshift=-2mm] at (A.west) {\rotatebox{90}{Garbage collected}};
  \node[anchor=east,xshift=-1cm,align=right] at (A.south west) {\rotatebox{90}{Memory management}};
  \node[anchor=east,xshift=-2mm] at (C.west) {\rotatebox{90}{Manual}};
  \node[anchor=north,yshift=-2mm] at (C.south) {No};
  \node[anchor=north,yshift=-1cm] at (C.south east) {Separate phase for compiling};
  \node[anchor=north,yshift=-2mm] at (D.south) {Yes};
\end{tikzpicture}

  \caption{Categorization of programming languages.}
  \label{fig:background:lang:categorization}
\end{figure}

\subsection{\idxx{Parsing}}

% file content
The contents of a file is a sequence of \defi{bytes}{Byte}, i.e., units that each have one of 256 possible values. When we write text, these bytes are interpreted as characters and the programs we use to open such files choose to draw each character to the right of the previous one. At \idx{line breaks}{Line break} -- represented by a specific sequence of characters -- the horizontal position is moved all the way to the left and the vertical position is moved one line height down. Some programs break lines that are too wide for your screen. But that's about it. Such a text file does not contain any information about font or text size. If you want to store such information, you have to introduce a \idx{file format}{File format} that \idx{encodes}{Encoding} such information by special \idx{markup}{Markup}. But this is not relevant for files that contain code.

% human interpretation
Files containing code are raw \idx{text files}{Text file}: A sequence of characters with line breaks. As programmers, we manipulate these files through a \idx{text editor}{Text editor}, which most often -- in addition to displaying these characters -- chooses to \idx{color}{Color}-code selected subsequences to make the text more readable for \idx{humans}{Human}. However, these colors represent an \textsl{interpretation} that the tool uses to enrich the code that is actually stored. Modifying the file and saving it will not cause the interpretation to be stored, but only the underlying code. Similarly, there are some such editors that number the lines, highlight a current line, and mark matching parentheses.

% machine interpretation
The text these files contain is called \textsl{\idx{program code}{Program code}} (or simply \textsl{code}), and we change it in much the same way as we change a written report. But when we ask the computer to \idx{execute}{Execute} our program code, it sees the file in a different way. The content is of course the same, but the \textsl{interpretation} is different: Based on some rules, the computer builds a \idx{tree}{Tree} structure of the code.

% TODO: tree structure

\subsubsection{Rules}

% https://homepage.ruhr-uni-bochum.de/jan.holthuis/posts/using-the-latex-rail-package

\subsubsection{Parse Trees}

% https://tex.stackexchange.com/questions/111564/create-a-syntax-tree-with-latex
