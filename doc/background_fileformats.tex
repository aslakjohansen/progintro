\section{File Formats}

% intro
Data is stored in files and they are organized in directories on in a filesystem. Files are organized in more or less \idx{standardized}{Standardization} formats. Any program that understands a format can work on files that follow this format. Some formats can be used for different purposes. For instance, a PDF file can be used for \idx{presentations}{Presentation} as well as \idx{reports}{Report}, and many different programs can produce such files. Figure \ref{fig:bg:fileformats} lists combinations of programs, the files they can manipulate, and the different uses of these files. Be careful not to mix these three concepts (\idx{program}{Program}, \idx{file format}{File!Format}, purpose): A presentation is only a \textsl{powerpoint} if it is in a \filename{ppt} or \filename{pptx} file format, and if someone requests a PDF presentation then a powerpoint simply won't do!

\begin{figure}[tbp]
  \section{File Formats}

% intro
Data is stored in files and they are organized in directories on in a filesystem. 

\begin{figure}[tbp]
  \section{File Formats}

% intro
Data is stored in files and they are organized in directories on in a filesystem. 

\begin{figure}[tbp]
  \section{File Formats}

% intro
Data is stored in files and they are organized in directories on in a filesystem. 

\begin{figure}[tbp]
  \input{figs/background_fileformats.tex}
  \caption{The relationship between programs, file formats and functions.}
  \label{fig:bg:fileformats}
\end{figure}

  \caption{The relationship between programs, file formats and functions.}
  \label{fig:bg:fileformats}
\end{figure}

  \caption{The relationship between programs, file formats and functions.}
  \label{fig:bg:fileformats}
\end{figure}

  \caption{The relationship between programs, file formats and functions.}
  \label{fig:bg:fileformats}
\end{figure}
