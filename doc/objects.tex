\chapter{Objects}

% working with structs

% example: reference as parameter to functions associated with a struct

% the struct is a concext: one context for each object

\section{Syntactic Sugar}

% methods rather than functions

% the constructor

\section{The Static Confusion}

% one context shared between all objects

% why call it a class?

\csharpsection{\csharp}
hello

\pythonsection{Python}
hello

\elixirsection{Elixir}
hello

\section{Case: Matrices}

% motivation: matrices are used in many forms of programming, graphics examples
In this section we will briefly cover \idx{matrices}{Matrix}. There are a number of domains where the ability for us to work with matrices is crucial. While they don't provide us with new functionality, they do represent a convenient \idx{abstraction}{Abstraction}, and they certainly help speed up \idx{graphics}{Graphics} and \idx{graph}{Graph} calculations.

% matrix def: rectangular array of numbers, numbers are called entries of the matrix, conventions (m*n matrix A: element names)
So what are they? They are rectangular \idx{arrays}{Array} of numbers. Each of these numbers are called entries of the matrix. The size of a matrix is named after the size along each of the two dimensions: A $2 \times 3$ matrix has a height of $2$ and a width of $3$. If we name the matrix $\mathbf{A}$, then the entries are called:
\begin{equation*}
  \mathbf{A} =
  \left[
    \begin{matrix}
      a_{11} & a_{12} & a_{13} \\
      a_{21} & a_{22} & a_{23}
    \end{matrix}
  \right]
\end{equation*}
So, the dimensions of a matrix are 1-indexed. Madness!

% operations: addition, subtraction
A number of operations are defined on matrices. One can add two matrices of the same size, and one can subtract one matrix from another of the same size:

% figure: addition and subtraction of matrices
\begin{align*}
  \mathbf{A}+\mathbf{B} &=
  \left[
    \begin{matrix}
      a_{11} & a_{12} & \cdots & a_{1n} \\
      a_{21} & a_{22} & \cdots & a_{2n} \\
      \vdots & \vdots & \ddots & \vdots \\
      a_{m1} & a_{m2} & \cdots & a_{mn} \\
    \end{matrix}
  \right]
  +
  \left[
    \begin{matrix}
      b_{11} & b_{12} & \cdots & b_{1n} \\
      b_{21} & b_{22} & \cdots & b_{2n} \\
      \vdots & \vdots & \ddots & \vdots \\
      b_{m1} & b_{m2} & \cdots & b_{mn} \\
    \end{matrix}
  \right]
  \\
  &=
  \left[
    \begin{matrix}
      a_{11}+b_{11} & a_{12}+b_{12} & \cdots & a_{1n}+b_{1n} \\
      a_{21}+b_{21} & a_{22}+b_{22} & \cdots & a_{2n}+b_{2n} \\
      \vdots & \vdots & \ddots & \vdots \\
      a_{m1}+b_{m1} & a_{m2}+b_{m2} & \cdots & a_{mn}+b_{mn} \\
    \end{matrix}
  \right]
  \\
  \mathbf{A}-\mathbf{B} &=
  \left[
    \begin{matrix}
      a_{11} & a_{12} & \cdots & a_{1n} \\
      a_{21} & a_{22} & \cdots & a_{2n} \\
      \vdots & \vdots & \ddots & \vdots \\
      a_{m1} & a_{m2} & \cdots & a_{mn} \\
    \end{matrix}
  \right]
  -
  \left[
    \begin{matrix}
      b_{11} & b_{12} & \cdots & b_{1n} \\
      b_{21} & b_{22} & \cdots & b_{2n} \\
      \vdots & \vdots & \ddots & \vdots \\
      b_{m1} & b_{m2} & \cdots & b_{mn} \\
    \end{matrix}
  \right]
  \\
  &=
  \left[
    \begin{matrix}
      a_{11}-b_{11} & a_{12}-b_{12} & \cdots & a_{1n}-b_{1n} \\
      a_{21}-b_{21} & a_{22}-b_{22} & \cdots & a_{2n}-b_{2n} \\
      \vdots & \vdots & \ddots & \vdots \\
      a_{m1}-b_{m1} & a_{m2}-b_{m2} & \cdots & a_{mn}-b_{mn} \\
    \end{matrix}
  \right]
\end{align*}

% operations: (matrix) multiplication
Matrices can take part in scalar multiplication and matrix multiplication. We will skip scalar multiplication in this book, and focus on matrix multiplication.

% figure: (matrix) multiplication

% special matrices: zeroes, ones, identity
On top of this, there are a number of special matrix configurations that we often use. Somethimes we need a matrix of a particular size where all entries are zero or one. At other times we need one that is square and has a diagonal of ones starting in the upper left corner. The rest of the entries in this matrix are zeroes. This matrix is called an identity matrix.

\csharpsubsection{\csharp}

% the part that is solved by the object

% the part that is not

\begin{figure}[tbp]
  \inputminted[fontsize=\footnotesize]{csharp}{../src/csharp/matrix/Matrix.cs}
  \caption{Implementation of \classname{Matrix} class.}
  \label{fig:objects:matrix:lib}
\end{figure}

\exercises{objects}{Objects}

