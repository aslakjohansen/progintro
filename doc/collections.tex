\chapter{Collections}

\section{Dynamic Arrays}

% problem: things were simple when we worked with months (always the same number), the number of items is not always known, shopping cart example, we could allocate space for the largest conceivable number of items, Amazon is estimated to have 600+ million products in their inventory (2025 number) and 315 million customers (2024), assuming that 1B is enough store how many products a customer orders (limit of 255) this gives us approx 190PB just to hold the shopping carts, this is not feasible, a different approach is needed

% the growth and recession approach

% algorithm

% figure

% models of growth rate

\subsection{Implementation}

\csharpsubsection{\csharp}

\section{Linked Lists}
\csharpsubsection{\csharp}
\elixirsubsection{Elixir}

\section{Maps}

% etymology: maps of the world, no perfect fit for a (mostly) spherical world, projections of physical features of the globe to a flat piece of paper (or screen), this we call a mapping from A to B, mappings exists everywhere in software, example of a shopper and a shopping cart

% figure: the circles illustration

% no natural representation

% hacking our way around it: array of pairs

% terminology: associative array (same), dictionary (mostly same), hash table (concrete implementation of a map)

\subsection{Hash Tables}

\subsubsection{Hashing}

\subsubsection{Organization}

\csharpsubsection{\csharp}
\elixirsubsection{Elixir}

\section{Sets}

\subsection{Set Operations}

\csharpsubsection{\csharp}

% interface+class combo

% destructive operations: what it means?, why destructive?

\elixirsubsection{Elixir}

