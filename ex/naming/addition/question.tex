Write a program following this recipe:
\begin{itemize}
  \item Create a new project with a class called \typename{Adder}.
  \item Create a static method named \methodname{Solve} that takes two \typename{int} parameters and returns a \typename{String}. This method must construct (and return) a \typename{String} that expresses what the sum of the two parameters is. For the inputs 3 and 5, this could be \say{\texttt{3 + 5 = 8}}.
  \item Create another static method by the same name that takes two \typename{double} parameters and returns a \typename{String}. This method, just like the first on, constructs its string in such a way that it expresses what the sum of the two parameters is. This could be \say{\texttt{3.14 + 5.12 = 8.26}}.
  \item Create a \methodname{Main} method which:
    \begin{enumerate}
      \item Calls \methodname{Solve} with two \typename{int} values and prints out the result to the screen.
      \item Calls \methodname{Solve} with two \typename{double} values and prints out the result to the screen.
    \end{enumerate}
  \item Compile and execute the program. Verify that the result is correct.
  \item What happens if we try to call \methodname{Solve} with an \typename{int} and a \typename{double}, and why does that happen?
\end{itemize}
