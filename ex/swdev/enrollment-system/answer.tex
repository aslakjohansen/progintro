The new program specification with noun-verb analysis:

\begin{center}
  \textsl{The \hlnoun{enrollment} system \hlverb{contains} a \hlnoun{list} of \hlnoun{students} and \hlnoun{courses}. \hlnoun{It} can \hlverb{display} the \hlnoun{list} of \hlnoun{students} which \hlverb{are} enrolled in a particular \hlnoun{course}. It is possible to \hlverb{enroll} \hlnoun{students} to a \hlnoun{course} and \hlverb{remove} \hlnoun{them} from a \hlnoun{course}. It is possible to \hlverb{add} and \hlverb{remove}, both \hlnoun{students} and \hlnoun{courses}, to/from the \hlnoun{enrollment} system.}
\end{center}

The new requirement specification:

\begin{itemize}
  \descitem{Functional requirements:} \\
    \begin{tabular}{|p{8mm}|p{4cm}|p{6cm}|}
      \hline
      \emph{ID} & \emph{Name} & \emph{Description} \\
      \hline
      F01 & Display a list of students & It must be possible to display a list of students \\
      \hline
      F02 & Display a list of courses & It must be possible to display a list of courses \\
      \hline
      F03 & Add and remove association & It must be possible to associate students to a course, and to remove such associations \\
      \hline
      F04 & Add and remove students & It must be possible to add students to and remove students from the system \\
      \hline
      F05 & Add and remove courses & It must be possible to add students to and remove courses from the system \\
      \hline
    \end{tabular}
  \descitem{Non-functional requirements:} \\
    \begin{tabular}{|p{8mm}|p{4cm}|p{6cm}|}
      \hline
      \emph{ID} & \emph{Name} & \emph{Description} \\
      \hline
      NF01 & Capacity & The system must be able to handle 10 courses \\
      \hline
      NF02 & Interoperability & The system must be able to integrate with itslearning \\
      \hline
    \end{tabular}
\end{itemize}

This leads to the new CRC card:

\begin{tabular}{|p{0.7\textwidth-2\tabcolsep}|p{0.3\textwidth-2\tabcolsep}|}
  \hline
  \multicolumn{2}{|l|}{\cellcolor{black!20} EnrollmentSystem} \\
  \hline
  \emph{\cellcolor{black!10} Responsibilities} & \emph{\cellcolor{black!10} Collaborators} \\
  \hline
  \begin{itemize}
    \item Contains list of \typename{Course}-objects
    \item Contains list of \typename{Student}-objects
    \item Can enroll \typename{Student}-objects in Course-objects
    \item Can remove \typename{Student}-objects from Course-objects
    \item Can display \typename{Student}-objects enrolled in any given \typename{Course}-object
    \item Can add \typename{Student}-objects to the list of \typename{Student}-objects
    \item Can remove \typename{Student}-objects from the list of \typename{Student}-objects
    \item Can add \typename{Course}-objects to the list of \typename{Course}-objects
    \item Can remove \typename{Course}-objects from the list of \typename{Course}-objects
  \end{itemize}
  &
  \begin{itemize}
    \item Student
    \item Course
  \end{itemize}
  \\
  \hline
\end{tabular}

Based on this we get the following UML class diagram:

\begin{tikzpicture}%[show  background  grid]
  \begin{class}[text width=6cm]{EnrollmentSystem}{0,0}
    \attribute{+ students: Student[]}
    \attribute{+ courses: Course[]}
    \operation{+ Enroll(Student, Course): void}
    \operation{+ Remove(Student, Course): void}
    \operation{+ ShowParticipants(Course): void}
    \operation{+ GetCourses(): void}
    \operation{+ GetStudents(): void}
    \operation{+ AddStudent(Student): void}
    \operation{+ RemoveStudent(Student): void}
    \operation{+ AddCourse(Course): void}
    \operation{+ RemoveCourse(Course): void}
  \end{class}
  
  \begin{class}{Course}{-4,-6}
    \attribute{name : string}
    \attribute{participants : Student[]}
    \operation{+ Enroll(Student): void}
    \operation{+ Remove(Student): void}
    \operation{+ GetParticipants(): Student[]}
  \end{class}
  
  \begin{class}{Student}{4,-6}
    \attribute{name : string}
    \attribute{id : int}
    \operation{+ GetName(): string}
  \end{class}
  
  \unidirectionalAssociation{EnrollmentSystem}{}{}{Course}
  \unidirectionalAssociation{EnrollmentSystem}{}{}{Student}
  \draw[umlcd style, fill=none, ->] (Course.east) |- node[above ,sloped , black]{} (Student.west);
\end{tikzpicture}

In the final code, only the \typename{EnrollmentSystem} class has been modified:

\inputminted{csharp}{\context/answer/EnrollmentSystem.cs}

