Write a program in which:
\begin{enumerate}
  \item Some number of seconds since a specific point in time (e.g., January 1st, 1970) is stored in a variable. Given an agreement of this starting point (January 1st, 1970), such a number can be used to represent a timestamp.
  \item Convert this number to an even number of years (lets assume that there are 365 days in every year) and how many days the timestamp is into this year.
    \begin{itemize}
      \item This way, we can convert the timestamp into something that makes more sense for humans: A combination of a year and a day. The day will then have a values between 0 and 364 (both included).
      \item Ideally we would add months as well so that we have a year-month-day date. Months, however, don't have the same lenght, and we don't want to include that complexity in this exercise.
      \item Added together, the two numbers (year and day) should be within 24 hours of our timestamp.
    \end{itemize}
  \item Print out the resulting year and day.
\end{enumerate}

Verify that the program works correctly.

