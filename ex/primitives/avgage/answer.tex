When executed the program prints out the following:

\begin{verbatim}
Average livespan of a male computer scientist: 76.666664
Average livespan of a female computer scientist: 68.666664
Average livespan of a computer scientist: 72.666664
Males lives this much longer than female: 8.0
\end{verbatim}

The program consists of (i) 6 variable declarations with initial assignment of values, (ii) a number of variable declarations where each variable is initialized to hold the result of some calculations, and (iii) printouts of the resulting values of these variable.

In the first part, six \typename{int} variables are declared. Each of these are named after a well-known computer scientist, and is initialized to their lifespan in years. For each of these a commented out link to their page on Wikipedia is provided.

Next comes four calculations, each of which is stored in a \typename{float} variable.

In the first calculation, all the \say{male} variables are summed up. The result of this is cast to a \typename{float} and then divided by three (i.e., the number of people). Finally the result is stored in the \varname{male\_avg} variable. In similar fashion, the average for the female computer scientists is calculated and stored in the \varname{female\_avg} variable. In the third calculation, these two averages are summed, then divided by two, and finally stored in the \varname{avg} variable. This is correct as the number of males and females are the same. As \varname{male\_avg} is a \typename{float}, it is not necessary to cast in order to avoid a potential rounding error in the division. In the last calculation, \varname{female\_avg} is subtracted from \varname{male\_avg} and the result is stored in the variable \varname{diff}.

Finally, each of these 4 variables are printed to the screen. To do this, two statements are used for each. In the first statement, a string that describes what the contents of the variable represents is printed without newline. In the second and last statement, the value is printed with newline.
