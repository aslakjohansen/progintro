In order to convert between \typename{int} and \typename{long} values we have to perform a cast. But, when do we have to do this explicitly, and when can it be implicit?

\begin{itemize}
  \item How are \typename{int} and \typename{long} different?
  \item Write a program that (i) declares \varname{i} as an \typename{int} variable, (ii) assigns it a value, (iii) declares \varname{l} as a \typename{long} variable, (iv) assigns the value of \varname{i} to \varname{l}, and (v) assigns the value of \varname{l} to \varname{i}.
  \item Determine experimentally when it is necessary to have explicit casts.
  \item Do the same thing with a \typename{float} variable \varname{f} and a \typename{double} variable \varname{d}.
  \item Experiment once again in order to find out where it is necessary to have explicit casts.
  \item Do the results of these experiments depend on the value you initially assigned to \varname{i}/\varname{f}?
\end{itemize}
