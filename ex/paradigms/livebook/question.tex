In this exercise, you will install Livebook\footnote{\url{https://livebook.dev}} on your laptop. The installation instructions are here: \url{https://livebook.dev/#install}.

If you are on Windows or OSX, you can install \textsl{Livebook Desktop}. If you are on Linux, you can install via Docker. Note that there are different ways to start Livebook through docker. Choose one that suits your preferences.

Once Livebook has been started, it exposes a web interface. As part of the startup process, it prints out the URL for this interface, and it is through this that we will work with Livebook. Point a browser to this URL and click under \say{Settings}. Under \say{Code editor} you will find a number of options, including:
\begin{itemize}
\descitem{Show completion list while typing} Make sure this is enabled.
\descitem{Show function signature while typing} Make sure this is enabled.
\descitem{Render ligatures} Make sure this is enabled.
\descitem{Wrap words in Markdown} Make sure this is enabled.
\end{itemize}
