\textbf{Delopgave 1}

To compile and execute the code:
\begin{minted}[fontsize=\footnotesize]{shell}
$ dotnet build
...
$ dotnet run
Unhandled exception. System.IndexOutOfRangeException: Index was outside the bounds of the array.
   at Program.<Main>$(String[] args) in .../ArrayIncrementer.cs:line 6
$ 
\end{minted}

\textbf{Delopgave 2}

In the printout of this run we can see that a \typename{System.IndexOutOfRangeException} exception is thrown.

\textbf{Delopgave 3}

The exception is thrown on line 6 of \typename{ArrayIncrementer}. On this line \varname{array} is being indexed using \varname{i} which at this point is bound to the value 5. As \varname{array} is initialized to hold 5 values the space allocated can only hold 5 values, and as \csharp\ is a 0-indexed language this means that the last valid index is 4. To avoid us trying to access memory outside of the \idx{allocation}{Memory!Allocation} of the \idx{process}{Process}, the \csharp\ runtime throws this exception.


\textbf{Delopgave 4}

\inputminted{csharp}{\context/answer/ArrayIncrementer.cs}

\textbf{Delopgave 5}

No, this is not the correct solution. This is caused by a \idx{bug}{Bug} in the code. It is clear to us that this exception will be thrown every time the code is executed. A non-ugly solution will involve us not getting into a situation where this exception is thrown, and this can be done by \textsl{correcting} the code.

