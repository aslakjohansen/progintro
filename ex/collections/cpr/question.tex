\textbf{Subtask 1}

Implement in \csharp\ a class for this UML diagram.

\begin{center}
    \begin{tikzpicture}
      \begin{class}[text width=6cm]{Person}{0,0}
        \attribute{- name: String}
        \attribute{- age: int}
        \attribute{- cpr: String}
        \operation{+ GetName(): String}
        \operation{+ GetAge(): int}
        \operation{+ GetCpr(): String}
        \operation{+ ToString(): String}
      \end{class}
    \end{tikzpicture}
\end{center}

\textbf{Note:} The \varname{cpr} field refers to a Danish social security number format called CPR.

\textbf{Hint:} When the \methodname{ToString} method is defined in the UML class, then you have to override the method and give it a meaningful implementation.

\textbf{Subtask 2}

Make a \typename{List} that contains 5 \typename{Person} objects. You decide the attribute values. One person must have the CPR number \texttt{010101-0101}.

\textbf{Subtask 3}

Loop through your \typename{List} object and find the person with the CPR number \texttt{010101-0101} by calling the method \methodname{GetCpr} on each object.

Print the result of an (explicit or implicit) call to the \methodname{ToString} method on this object.

\textbf{Subtask 4}

Create a \typename{Dictionary<string, Person>} map and add the \typename{Person} objects to this map that you previously added to the \typename{List} object in subtask 2. Use the CPR number as the \textsl{key}.

\textbf{Subtask 5}

Find the person with the CPR number \texttt{010101-0101} in the \typename{Dictionary} object and print the value returned from a call to the \methodname{ToString} method on this object.
