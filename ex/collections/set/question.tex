From highschool, you know the following set operations:

% inspiration: https://texample.net/tikz/examples/set-operations-illustrated-with-venn-diagrams/
\begin{center}
\begin{tikzpicture}
  \newcommand{\firstcircle}[2]{(#1,#2) circle (1.1cm)}
  \newcommand{\secondcircle}[2]{(#1+1.6cm,#2) circle (1.1cm)}
  \newcommand{\xoffset}[0]{4.1cm}
  
%  \def\firstcircle{(0,0) circle (1.5cm)}
%  \def\secondcircle{(0:2cm) circle (1.5cm)}
  
  \colorlet{circle edge}{blue!50}
  \colorlet{circle area}{blue!20}
  
  \tikzset{filled/.style={fill=circle area, draw=circle edge, thick},
    outline/.style={draw=circle edge, thick}}
  
  % intersection: A and B
  \begin{scope}
    \clip \firstcircle{0*\xoffset}{0};
    \fill[filled] \secondcircle{0*\xoffset}{0};
  \end{scope}
  \draw[outline] \firstcircle{0*\xoffset}{0} node {$A$};
  \draw[outline] \secondcircle{0*\xoffset}{0} node {$B$};
  \node[anchor=south] at (0*\xoffset+1.6cm/2, 1.1) {Intersection};
  
  % union: A or B
  \draw[filled] \firstcircle{1*\xoffset}{0} node {$A$}
                \secondcircle{1*\xoffset}{0} node {$B$};
  \node[anchor=south] at (1*\xoffset+1.6cm/2, 1.1) {Union};
  
  % difference: B but not A
  \begin{scope}
    \clip \firstcircle{2*\xoffset}{0};
    \draw[filled, even odd rule] \secondcircle{2*\xoffset}{0}
                                 \firstcircle{2*\xoffset}{0};
  \end{scope}
  \draw[outline] \firstcircle{2*\xoffset}{0} node {$A$}
                 \secondcircle{2*\xoffset}{0} node {$B$};
  \node[anchor=south] at (2*\xoffset+1.6cm/2, 1.1) {Difference};
\end{tikzpicture}
\end{center}

These operations can be implemented using the instance methods provided by \typename{Set}. These operations, however, are \idx{destructive}{Operation!Destructive} and that's often not what you want. Lets make a different variant.

\textbf{Subtask 1}

First, declare a \typename{NondestructiveSet<T>} interface (i.e., a parameterized \typename{NondestructiveSet}) with the methods:
\begin{itemize}
  \item \methodname{Intersection} The intersection contains all elements that are in both sets.
  \item \methodname{Union} The union contains all elements that are in at least one of the sets.
  \item \methodname{Difference} The difference contains all elements from one set that are not in another set.
\end{itemize}

These methods all take a \typename{NondestructiveSet<T>} parameter and return a new \typename{NondestructiveSet<T>}.

\textbf{Subtask 2}

Next, you need to implement a \typename{NondestructiveHashSet<T>} class that inherits from \typename{HashSet<T>} and implements \typename{NondestructiveSet<T>}. Make sure that none of the methods from \typename{NondestructiveSet<T>} affect their inputs.

\textbf{Subtask 3}

Demonstrate that your implementation works as expected.

\textbf{Subtask 4}

Extend your \typename{NondestructiveSet<T>} class with methods for:
\begin{itemize}
  \item \methodname{IsDisjoint} which returns \valuename{true} if and only if the \typename{NondestructiveSet<T>} argument it receives has no elements in common with itself.
  \item \methodname{IsSubset} which returns \valuename{true} if and only if all elements in its parameter (of type \typename{NondestructiveSet<T>}) exist in itself.
\end{itemize}

Implement these in \typename{NondestructiveHashSet<T>} and demonstrate correctness.
