In the following instructions you will find a number of URLs. These are specific to the github user \say{aslakjohansen}. Replace this part with your own user name.

As so many other websites, the interface of github is constantly changing. Because of this, you should expect to have to deviate from the following instructions:

\begin{itemize}
  \item Go to the page for your account on GitHub: \url{https://github.com/aslakjohansen/}.
  \item In the top right corner you will find a drop-down menu marked with a \say{+}. Click this and choose \say{New repository}. Name the repository \say{progintro-exercises} and make a decision about who should be able to see your work on this repository: public or private. This visibility can be changed later on. The remaining default values are fine. Click \say{Create repository}.
  \item You have now created a repository. Go to the page for it (\url{https://github.com/aslakjohansen/progintro-exercises/}). Click the green \say{<> Code} button, and choose \say{Local} followed by \say{SSH}. Here, you copy the URL (\url{git@github.com:aslakjohansen/progintro-exercises.git}).
  \item Now, you have to decide where in your computers filesystem you want to store the local copy of this repository. This directory can be moved later on.
  \item Open a terminal and change directory to this location. Run the git command for cloning the repository: \texttt{git clone git@github.com:aslakjohansen/student-projects.git}.
  \item Verify that a \filename{progintro-exercises} directory has been created.
\end{itemize}

