\textbf{Subtask 1}

Save the following as \filename{example.svg} and open the file in a browser:
\inputminted[fontsize=\footnotesize]{xml}{\context/question/example.svg}

This is an example of a file format called Scalable Vector Graphics (\idx{SVG}). It is the only format for vector graphics that is widely supported on websites. The above is just a small sample of what you can do in SVG.

\textbf{Subtask 2}

Try modifying the individual values ​​(but do not touch the reference to \url{http://www.w3.org/2000/svg}) to get an understanding of how the different elements affect the way the browser \textsl{draws} the file.

\textbf{Subtask 3}

Find your solution from the exercise in section \ref{q:functions:sudoku-print}, or -- alternatively -- start from the reference solution for it in section \ref{a:functions:sudoku-print}.

Your task is now to create a new variant of this code base that produces an SVG file instead of printing some characters to the screen. Make it look pretty!
