Create a program in which the user is presented with a prompt through which commands can be entered. More precisely:
\begin{enumerate}
  \item Create a \methodname{Main} method.
  \item In this method, create a \typename{bool} variable named \varname{done} and initialize it to \texttt{false}.
  \item Place a \texttt{while}-loop inside the \varname{Main} method and use as condition \texttt{!\varname{done}}. Try to pronounce this line!
  \item The rest of the code will be placed inside this \texttt{while} loop.
  \item First, a prompt \underline{without line breaks} is printed. This could be the text string \say{\texttt{\$ }}.
  \item Next, a line is read.
  \item Split this line on spaces.
  \item A reference to the resulting list is stored in a variable named \varname{command\_args}. What would a reasonable type for this be?
  \item A reference to the first element of \varname{command\_args} is stored in a variable named \varname{command}.
  \item Use a \texttt{switch} construct on \varname{command} to implement a series of commands. Implement at least the following commands:
    \begin{enumerate}
      \descitem{exit} This sets the variable \varname{done} to the value \texttt{true}.
      \descitem{echo} This prints the same text as it received.
      \descitem{time} This prints the current time.
    \end{enumerate}
  \item Make sure to print an informative error message if \varname{command} is not recognized.
\end{enumerate}

Test that the above functionality works.
